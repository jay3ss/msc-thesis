\documentclass[oneside,12pt]{book}
    % The option oneside is required for UNLV theses.
    % The 12pt is optional -- UNLV will allow 10pt too.
    % 10pt is the default, so to switch to 10pt, just delete 12pt in the above.
\usepackage{amsthm,amssymb,amsmath,graphicx}
\usepackage{url}
%\usepackage[toc,page]{appendix}
\usepackage{etex}
\usepackage{multirow}
\usepackage{graphicx,subcaption}
\usepackage[table,xcdraw]{xcolor}
\usepackage{listings,matlab-prettifier}
%\usepackage{graphicx}
\usepackage{pstricks,pst-node,pst-plot,pstricks-add,pst-grad,pst-slpe,overpic}
%\usepackage{showframe} - to show the frame layout
%\usepackage{layout} - to show the page layout
%\usepackage{tikz}
%\usetikzlibrary{arrows,shapes}
\usepackage{amsmath,mathtools,amsthm}
%\usepackage{epstopdf}
\usepackage{caption,verbatim}
%\usepackage[section]{placeins}
%
\newcommand{\sgn}{\mathop{\mathrm{sgn}}}
%\newtheorem{theorem}{Theorem}[section]
%\renewcommand{\thetheorem}{\arabic{section}.\arabic{theorem}}
%\newtheorem{lemma}[theorem]{Lemma}
%\newtheorem{proposition}[theorem]{Proposition}
%\newtheorem{corollary}[theorem]{Corollary}
%\newtheorem{conjecture}[theorem]{Conjecture}
%\newtheorem{definition}[theorem]{Definition}

 % Optional packages.
\usepackage{setspace,myCSULAthesis} % Required packages.  Look at UNLVthesis.sty to see how
    % LaTeX is instructed to set things up.  This file may need some tweaking.
    % setspace.sty is not normally part of MikTeX.  It can be obtained from www.ctan.org.
    % Do a search on setspace.sty.
    % The files setspace.sty and UNLVthesis.sty should be in the same directory
    % as this file (or in a directory of MikTeX where LaTeX will know to find it
    % -- for example, where other style files are).
    
\usepackage{hyperref}
%pagestyle{unlvplain}
\pagestyle{plain} %  This is defined in UNLVthesis.sty.  Headings are empty except for page numbers,
    % and the page numbers are the same size as the text.  Most documents use a different size so that
    % it it is difficult to mistake it as part of the text.

%These define the format and numbering of theorem like environments.
\newtheorem{theorem}{Theorem}
\newtheorem{corollary}[theorem]{Corollary}%Corollaries and Lemmas are numbered as theorems.
\newtheorem{lemma}[theorem]{Lemma}

%These define the format and numbering of definition like environments.
\theoremstyle{definition}%This environment is not in italics, like theorems are.
\newtheorem{definition}{Definition}
\newtheorem*{introduction}{Introduction}%The * means it is unnumbered.
\newtheorem*{conclusion}{Conclusion}

% Put definitions here.  For example, suppose you often use the Greek characters
% alpha, beta, etc., which in LaTeX are \alpha, \beta, etc. (in math mode only).
% Then it may be easier to create shortcuts for these commands, such as:
%\def\aa{\alpha}
%\def\bb{\beta}
% Now, instead of typing \alpha, we can type \aa.
% Here's another I often use:
\def\Bbb#1{{\mathbb #1}} % \Bbb is an obsolete command, but I'm old
% and still use it, so I define it to do what it used to do.  The usage
% is like \Bbb R, which will produce a blackboard bold R, and is literally
% translated to {\mathbb R}.  Note that this command includes a single
% argument.

\begin{document}
%\layout{} --- to check the layout margins of the page enable it!!
\pagenumbering{roman}% Items before Chapter one have roman numbers (if any).
\include{Titlepage} %Replace this file name with the name of your title page.
    % A copyright statement is optional and would be placed here.
    % The copyright page has no page number -- the title page is always page i and the
    % Thesis Approval page is always page ii.
\newpage \setcounter{page}{3} %The Thesis Approval page is page ii.  It is inserted separately.
\chapter*{ABSTRACT}
\addcontentsline{toc}{schapter}{ABSTRACT}% This command adds the Abstract to the table of contents.

\begin{center}
\textbf{Digital Media, Social Campaigns, and Fake News: Mathematical Modeling and Control Methods}

by

Michael Muhlmeyer

 $\langle$Dr. Shaurya Agarwal$\rangle$, Thesis Advisor\\*[-12pt]%Single spaced.
 Professor of Electrical Engineering \\*[-12pt]
 California State University, Los Angeles  %

 
 \end{center}

The role of information spread and the impact it has on societies in the modern world cannot be understated. In the age of mass communication, digital misinformation, and social media, the importance of understanding and developing control mechanisms for information spread are doubly necessary. While traditional information spread has been examined in detail from a variety of angles over the decades, little attention has been given to the relatively recent phenomena of the super-fast spread of information via social media and the rise and impact of ``fake news" within said information networks. In Part I, a background of information spread theory, terminology, and applications are presented and organized in both a general setting and specifically as information spread applies to social media networks. The importance and influence of a network's structure to the spread of information is also discussed. In Part II, several traditional dynamic models are presented, built upon, and re-framed in the modern context of social media information spread using differential equations. A new model is proposed to address networks and sets of adjacent networks in which information learned and spread is highly polarized, contentious, or unverified. In Part III, control mechanisms and strategies are examined and evaluated along with supporting social theory. These control strategies are developed and applied to sample case studies using the models discussed previously. The results from the proposed control methods of the sample scenarios are calculated, simulated, and discussed. % This is where the abstract goes.
\chapter*{ACKNOWLEDGEMENTS}
\addcontentsline{toc}{schapter}{ACKNOWLEDGEMENTS}

I would like to thank to L.A. Adamic, N. Glance, and Gallup for the use of their images in the spirit of academic advancement. Without these resources, my points would both less strong and accessible.

This work would	not	have been possible without the financial support of	California State University, Los Angeles and their generous academic grants and the support of the faculty and staff of the Electrical and Computer Engineering Department. I am especially indebted	to my professors during my time here who have been supportive	of my career goals and who have worked actively to help me to pursue those goals, as well as providing their extensive professional and personal guidance. I would also like to thank my thesis committee members for their input, advice, and interest in this work. 

I am grateful to all of	those with whom I have had the pleasure to work during this	and	other related projects.	I would especially like to thank Dr.~Shaurya Agarwal, the	chairman of	my committee. As my	teacher	and	mentor,	he has taught me more than I could ever give him credit for here. He has shown me, by his example, what a	good researcher, engineer (and person) should be.

Nobody has been	more important to me in	the	pursuit	of this	project	than the members	of	my	family.	I would	like to	thank	my	parents, whose love	and	guidance are with me in	whatever I pursue. They	are	the ultimate role models.	Most importantly, I wish to	thank my loving	and	supportive fiance, Liliane, who has	provided unending understanding, encouragement, and inspiration.  
\tableofcontents %inserts table of contents
\listoftables \addcontentsline{toc}{schapter}{\listtablename}%
\listoffigures \addcontentsline{toc}{schapter}{\listfigurename}% Comment these out if there are no figures or tables.

%Acknowledgements come after the tables.

\newpage % Do not remove this command.  It's there to make sure the page numbering is correct.
\pagenumbering{arabic} %Chapter 1 begins on page 1.

\part{INTRODUCTION}
\chapter{Motivation} \label{chapter_motivation}

\section{Why Information Spread Matters}
Humans communicate. We share ideas, technology, opinions, strategies, religion, art, and other cultural elements. In fact, spreading information is a large part of what separates humans from the vast majority of the animal kingdom. As such, information spread has always been an underlying part of not only an individual human experience, but also our progress as a species. Until relatively recently, information spread (or ``rumor" spread) was at best examined through intuition, experience, and assumptions. A disease outbreak in an early settlement could be communicated to neighboring communities through a messenger, but there was no way of knowing how far and to what degree that information would penetrate a targeted community. As writing became widespread, so too did the capacity of a community to spread information within itself and to its neighbors. Later, more advanced technologies such as the printing press, telegram, telephone, radio, and television added to the older information spread methods, allowing a small number of individuals to spread information or communicate to the masses with relative ease. Finally, in the modern information age, high speed near-instant communication is available to nearly every individual in developed societies. Why does our ability to spread information matter? Because the quality and capacity of information spread can have a deep impact on any given society. Election campaigns succeed when a candidate spreads their message and policies to as many voters as possible. Widespread advertisements result in greater purchases and profits than unadvertised products. Natural disasters and other emergencies can be more effectively and quickly dealt with by spreading the word of areas to avoid or precautions citizens should exercise. Countless examples exist. The point is that we rely on communication and information spread to function as human beings living in a modern society. As such, studying and understanding how different types of information spread or recede is highly valuable in several industries and social structures. With modern knowledge in areas such as sociology, mathematics, engineering and information science there are more tools available now than ever before to effectively understand, predict, and control information as it moves throughout a community.

\section{Modern Scenarios}
Consider the ``Salt Panic" in China. In March 2011, a tsunami following the Tohoku earthquake led to three nuclear meltdowns, explosions, and the release of radioactive material in Japan. Upon hearing the news of the disaster along with a false rumor that iodized salt could help prevent radiation poisoning, panicked shoppers stripped Bejing stores of salt. As a result, salt prices were said to have increased up to 10-fold in some areas. The Chinese government and international scientists repeatedly announced that there was no reasonable threat from the radiation. Even in the event of dangerous radiation reaching China, the basic table salt found in stores would not help mitigate any radiation effects. Eventually, with efforts from local governments, the false rumors were eventually quelled. Regardless, the Salt Panic demonstrates the power and virility of mass information spread, be it true or false. 

Consider a second example. Amazon, the popular online shopping site, has progressively gained better and better insight into patron purchase habits. By tracking what customers view and have purchased the past, they offer targeted recommendations. In addition to this, many users actively post reviews of products and seek out reviews for potential products they are considering for purchase. While the first element is simply accomplished via machine algorithm and data acquisition, the second is a direct result of active information spread within an online shopping community. While a product may advertise to entice customers to buy a product, reviewers may give positive or negative feedback and ratings, which could have widespread influence over the general positive or negative perception and value of the product. Clearly, information spread is an integral part of the modern online shopping experience, which cannot be ignored.

\section{Campaigning}
Of particular interest in the study and application of information spread is the concept of campaigning. In a campaign, measures are taken to deliberately spread a message throughout a population. Often, information spread campaigns are seen in the form of advertising campaigns for products or services and political campaigns for candidates or propositions. Effective spreading of information can be especially powerful in these areas because the initial spreader is both creating information (true or not) and pouring resources into spreading it until the message hopefully gains enough traction and widespread belief that profit is returned to the original spreader. For advertisers, this means common knowledge of the product or service becomes popular, sells well, and brings financial benefit. For politicians, voters become aware of the candidate's highlighted promises, ideology, and qualifications (or negative attributes of political rivals) to ultimately gain votes. Both product advertisers and politicians pour vast amounts of capital into campaigns for a reason: it works and is, in fact, believed to be required for large scale success over competitors. 

Just as effective as building up a candidate or product, information spread can be utilized to tear down opponents. Countless smear campaigns are riddled throughout history. One need not look further than old election attacks by founding fathers of the United States, Thomas Jefferson and John Adams. During election campaigning, Jefferson's hired attacker accused President Adams of having a ``hideous hermaphroditical character, which has neither the force and firmness of a man, nor the gentleness and sensibility of a woman." Adams' men called Vice President Jefferson ``a mean-spirited, low-lived fellow, the son of a half-breed Indian squaw, sired by a Virginia mulatto father." Adams was labeled a fool, a hypocrite, a criminal, and a tyrant, while Jefferson was branded a weakling, an atheist, a libertine, and a coward \cite{cnnffcampaign}. The idea of actively spreading information proved to be incredibly effective in democratic politics. Due in no small part to Jefferson's hired ``hatchet man", he was able to win the first hotly contested Presidential election in the United States. 

In corporate and product advertising campaigns, many may recall the popular ``Get a Mac" television and internet campaign by Apple in 2006, in which personifications of Macintosh and Windows PCs introduce themselves as ``I'm a Mac" and ``I'm a PC" and proceed to act out various skits aimed at touting the benefits of a Macintosh over a Windows PC. The campaign was massively successful and gained popularity and recognition worldwide, leading to a thirty-nine percent increase in Macintosh computer sales that year. Microsoft eventually released similar ads meant to parody and similarly appear superior to their competitors with nominal success \cite{getamac2016}. 

\section{Fake News}
The concept of ``fake news" has recently become a hot topic in sociopolitical discussion, particularly during and immediately following the 2016 U.S. Presidential Election. That said, the notion of spreading a fake story or lie to further one's cause is hardly new. Returning to the 1800 United States Presidential Election. Adams' loss of the election was in no small part due to the effectiveness of the smear campaigning, but also by the application of fake news via a deliberately false story that Adams wanted to go to war with France \cite{cnnffcampaign}. Similar examples of untrue stories framed as news can be seen throughout United States and world history, particularly when the views and opinions of the populace are important (as they are in many democratically governed nations).

One must take care in differentiating between fake news and simply untrue rumors. While ``hearing" that a celebrity has died and spreading that information (while the celebrity is in fact alive and well) may be untrue, it is not being framed and presented in such a way as to be viewed by a typical reader as an official and verified true news story. Similarly, tabloid magazines with unverified stories are rarely perceived as reliable news sources and in effect are collected (true or untrue) rumors. In order for there to be actual fake news, the story must be knowingly false to the initial spreader and framed in such a way as to seem like legitimate news. 

Fake news can take many forms, from independent internet news sites to Facebook fan pages to WhatsApp sponsored messages \cite{whatsapp2017}. While the origin of fake news can start from any number of sources, it solidifies itself as appearing to be legitimate news once the story is picked up (and likely never properly fact-checked) by a widely read news source. At this point, the fake news can spread quickly and effectively in ways similar to legitimate news. 

One major challenge currently is how to quickly identify and mitigate the proliferation of fake news. With the advent of social media as a source of accepted news, both verified and fake news spreads quickly along social networks (which can mean global spread in some instances) with little to no vetting. In fact, in modern news cycles, there is great pressure to release news as fast as possible, oftentimes circumventing traditional news journalism fact checking procedures. As a result, fake news is more easily absorbed into and spread throughout the public consciousness as legitimate news, true or not. Because of this, new strategies must be developed to either mitigate or proliferate fake news (depending on one's goals) to keep pace with the modern digital age of information spread. %Replace these with your chapters.
\include{Chapter_Background} %Background Section
%\chapter{CASE STUDIES}% Chapter titles are capitalized

\section{Case Study 1}

\section{Case Study 2}

\section{Case Study 3}
 %Case studies to be examined throughout
\chapter{Social Networks and Digital Media}% Chapter titles are capitalized
\section{Social Media Network Theory}
Several theories exist in an attempt to describe and contribute to our understanding of social media. While some theories go about examining the role each individual plays in social media information spread, others take a higher level view and focus on the dynamics of online information and how it is communicated with and between groups as an abstract entity.

\subsection{Word of Mouth Theory}
In word of mouth theory, research finds that individuals are far more likely to consume a product, for example, if advice and information about it comes from friends and relatives \cite{crotts1999consumer}. Generally speaking, this is because the information from those sources is viewed as being more credible and trustworthy than a simple advertisement from an outsider, which is essentially a paid recommendation on a product or service. With the advent of social media online reviews of products, services, or experiences, there is no shortage of opinions and recommendations from strangers who hold little to gain by giving a positive or negative review \cite{sigala2012social}.

\subsection{Social Exchange Theory}
Fundamental to the existence of online social media is the requirement that individuals or groups create and communicate content, else there would be no information to spread \cite{sigala2012social}. Social exchange theory emerged from sociology studies which sought to examine the exchange relationships between individuals and small groups \cite{emerson1976social}. In social exchange theory, individuals act in accordance with a subconscious cost-benefit analysis type mentality, subjective to the individual. If a social behavior is deemed too costly, such as insulting another community member, it will not be acted upon unless there is a greater perceived benefit (perhaps in this case, a benefit of ascerting social dominance). Cost-benefit mentality influences our ability to communicate, form bonds with members of a community, and spread information within the community \cite{emerson1976social}. Homans summarizes the theory of social exchange well, by writing:

\begin{quote}
Social behavior is an exchange of goods, material goods but also non-material ones, such as the symbols of approval or prestige. Persons that give much to others try to get much from them, and persons that get much from others are under pressure to give much to them. This process of influence tends to work out at equilibrium to a balance in the exchanges. For a person in an exchange, what he gives may be a cost to him, just as what he gets may be a reward, and his behavior changes less as the difference of the two, profit, tends to a maximum.  \cite{homans1958social}
\end{quote}

Clearly, in a social media driven community, individuals expect to give and gain reputation and influence in the abstract sense via posts, comments, shares, and other popular mechanisms. Expected rewards from these social exchanges may not be monetary, but they certainly can be in the form of sponsored advertisements on social media. 

It is also noteworthy that in an online social media environment such as YouTube, far more individuals are consuming content over those creating it. According to the Global Web Index 2009 study on online social media habits in the United States \cite{trendstream2010}, online social media users can be categogized as belonging to four main groups: watchers, sharers, commenters, and producers. Watchers ($79.8\%$) view and follow online content only, with no reciprocation. Sharers ($61.2\%$) share, upload, or otherwise spread the content of others. Commenters ($36.2\%$) are individuals who will rate, review, and comment on things like products as a form of contribution without actual material generation. Finally, producers ($24.2\%$) create their own content for any number of reasons, ranging from expression to social recognition \cite{sigala2012social}. The validity of the groupings requires additional research, but it certainly provides some insight into online social media behavior. That said, promising research has been done to demonstrate affinity, belonging, interactivity, and innovativeness are all included in the base expectations of users when utilizing a social media network \cite{krishen2016generation}.

\subsection{Social Network Analysis}
In social network analysis, each community member is treated as a node and their communication with other members is treated as a link or connection between nodes. Social networks are analyzed at varying scales, but the main purpose of social network analysis is to ultimately utilize mathematical models to study the structure, development, and evolution of the social network \cite{wasserman1994social}. Crucial to social network analysis is its structure, as that will dictate the efficiency by which information can be spread throughout a social network group. The advantage of a social network analysis method of examining social media groups is its ability to quantify relationships mathematically. Social network analysis preliminaries are discussed further in the next chapter.

\section{Social Networks}
Critical to the discussion of modern information spread is the concept and influence of social networks. First, let us define a general network. A network is a set of objects or nodes along with a mapping or description of the relationship between the nodes. \cite{kadushin2012understanding} A social network is then a set of individuals which are related in some way such that their relationship can be mapped or traced. 

Consider the most basic case of two friends who are linked to each other as shown in Figure \ref{fig:Symmetric_Relationship}. In this case, \textit{individuals}  $1$ and $2$ are linked within a social network. Specifically, they are symmetrically linked because each has two-way mutual communication with the other. In traditional human interactions, this is the most common social network mapping of a face-to-face communication between friends. Formally, this mapping of individuals is called a ``sociogram".
\begin{figure}[!htbp] \centering
  \includegraphics[width=0.7\linewidth]{drawings/Symmetric_Relationship.eps}
  \caption{Individuals $1$ and $2$ in a simple symmetric relationship}
  \label{fig:Symmetric_Relationship}
\end{figure}
In Figure \ref{fig:3_node_sociogram1}, a third individual is added to the network. Notice that the new actor, \textit{individual-$3$} has a symmetric relationship with \textit{individual-$1$}, but is only singularly directional to \textit{individual-$2$}. Perhaps \textit{individual-$3$} is a writer. While \textit{individual-$2$} is being influenced by the information spread from \textit{individual-$3$} as he reads his work, \textit{individual-$3$} has no direct knowledge of or contact with \textit{individual-$2$}, while \textit{individual-$1$} is acquainted with and talks to both of the remaining people. 
\begin{figure}[!htbp] \centering
  \includegraphics[width=0.7\linewidth]{drawings/3_node_sociogram.eps}
  \caption{A three-node relationship mapping of individuals $1, 2,$ and $3$}
  \label{fig:3_node_sociogram1}
\end{figure}
What happens if a fourth individual who is only acquainted with \textit{individual-$2$} enters the network map as shown in Figure \ref{fig:Intermediary_Relationship}? While \textit{individual-$4$} and \textit{individual-$3$} are symmetrically linked, she has no direct relationship to others in the network. She is said to have an ``intermediary relationship" to the rest of the network. In this case, the intermediary is \textit{individual-$2$} who serves as her link to the rest of the social network.
\begin{figure}[!htbp] \centering
  \includegraphics[width=0.7\linewidth]{drawings/Intermediary_Relationship.eps}
  \caption{Individual $2$ is an intermediary between $4$ and the rest of the network}
  \label{fig:Intermediary_Relationship}
\end{figure}
As more people are added to a social network, their interrelationships become increasingly complex. Notice the variety of symmetric and unidirectional relationships in Figure \ref{fig:Complex_Network}. By only adding a few more people to our social network, each with their own relationship links, the social network has grown complex enough that it proves difficult to trace and predict how information might spread between individuals on opposite sides of the network map. 
\begin{figure}[!htbp] \centering
  \includegraphics[width=0.7\linewidth]{drawings/Complex_Network.eps}
  \caption{A simplified complex social network}
  \label{fig:Complex_Network}
\end{figure}
In the next sections, we will address social networks in more detail in relation to their structure and formalized descriptive elements.

\section{Popular Social Networks}
%Twitter, Facebook, Linkedin, etc
When examining social networks in a practical and modern sense, discussion will typically be within the context of popular digital social networks that are used to absorb and spread information within a population. More than any other type of social network in the past, digital social networks have revolutionized the speed and reach of information spread. News, rumors, and advertisements can reach from one section of the globe to another in mere seconds. These digital (or online) social networks, are specifically designed to collect and form online communities and encourage the spread of information both within the group and between adjacent groups. 

\subsection{Twitter}
Twitter is a social networking site that allows users to post a short limited-character message over the internet via the Twitter website, a dedicated application, or a mobile device such as a cellphone. Twitter posters, or ``Tweeters", will often post a ``tweet" concerning what they are doing or thinking. The tweet is often accompanied by a reference tag known as a hashtag, that allows users to view similarly tagged tweets as a collection, usually referencing the same topic, event, or idea. Twitter is also used to post both pictures, news, and current events. Many view Twitter as a quick and easy way to find out what is happening around the world through by searching relevant keywords for news, trends, or current events.

Twitter is an especially popular social media platform for analyzing and collecting social network data because of it's direct and traceable nature. Hashtag trends and ``retweets" are relatively easy to collect and visualize compared to other online social media systems. Oftentimes, data science and machine learning algorithms for social research use Twitter as a primary data collection source.

\subsection{Facebook}
In contrast to Twitter, Facebook as a social media networking site focuses around each user's ``News Feed". The News Feed is a page customized for each individual user, which highlights and tracks the activities of their fellow community ``friends". Users make posts and friends of users can comment on or ``like" said posts as a sign of agreement or interest. The idea of photograph or image sharing is much more pronounced and integral to Facebook when compared to Twitter. On Facebook, each user's homepage is a collection of posts, discussions, and events based on the activities of that user's friend community. Given the more personal nature of Facebook, there have been several concerns over privacy issues as to where and how this posting, liking, photograph viewing, and commenting information is shared.

Facebook is often marketed as a way for real-life friends to stay in touch after separations due to distance, change of lifestyle, or other key dividing factors that would otherwise cause individuals to slowly lose track of one another. Due to the personal nature of Facebook, it has become a breeding ground of several forms of information spread, such as targeted advertisements, social movement growth, and news article distribution. Recently, Facebook has seen negative attention for its amount of user freedom afforded, leading to concerns over the spreading of hate speech and the proliferation of fake news \cite{facebookcrisis2016}. 

\subsection{LinkedIn}
While Twitter and Facebook are online social media networking sites catered towards news, opinionated commentary, and socializing, LinkedIn serves individuals and communities interested in employment-centered and professional networking. On LinkedIn, employers post job openings and company information pages, while job seekers set up professional profiles that include resumes, job experience, and CVs. Both employers and job seekers form ``connections" through the site to build a network of like-minded professionals to hopefully pair employers with prospective employees. Users can follow various companies, ``endorse" another user for a particular profile stated skill, post job listings, and more. Unlike many of the other social media sites, LinkedIn is not concerned with casual information spread, so much as advertisement (of one's self or one's company in this case).

Due largely to its narrowly defined nature, LinkedIn is not seen as controversial or particularly interesting in so far as large-scale or viral information spread is concerned, but it is an excellent example of a small scale focused advertising network. Additionally, it is mostly free from some of the complexities of more open online social networks, such as fake news, political agendas, and socio-cultural divisions. Most users are simply there to post their employment information and have it spread sufficiently to find a connection with a suitable employer toward the end of gaining a job. 

\begin{table}[]
\centering
\begin{tabular}{|l|l|l|l|}
\hline
                                                                              & \textbf{Twitter}                                                           & \textbf{Facebook}                                                                  & \textbf{LinkedIn}                                                                      \\ \hline
\textbf{Site Focus}                                                           & \begin{tabular}[c]{@{}l@{}}News, content,\\ story sharing\end{tabular}     & \begin{tabular}[c]{@{}l@{}}News, content,\\ story sharing\end{tabular}             & \begin{tabular}[c]{@{}l@{}}Company and industry\\ news and discussions\end{tabular}    \\ \hline
\textbf{\begin{tabular}[c]{@{}l@{}}Spreading \\ Mechanism\end{tabular}}       & \begin{tabular}[c]{@{}l@{}}tweets, re-tweets,\\ subscribing\end{tabular}   & \begin{tabular}[c]{@{}l@{}}Personal page,\\ comments, likes,\\ shares\end{tabular} & \begin{tabular}[c]{@{}l@{}}Company follows,\\ endorsements,\\ discussions\end{tabular} \\ \hline
\textbf{\begin{tabular}[c]{@{}l@{}}Outside \\ Impact\end{tabular}}            & \begin{tabular}[c]{@{}l@{}}Direct links from\\ posted content\end{tabular} & \begin{tabular}[c]{@{}l@{}}Direct links from\\ posted content\end{tabular}         & \begin{tabular}[c]{@{}l@{}}Direct links from\\ posted content\end{tabular}             \\ \hline
\textbf{\begin{tabular}[c]{@{}l@{}}Advertising \\ Opportunities\end{tabular}} & \begin{tabular}[c]{@{}l@{}}Promoted tweets\\ and trends\end{tabular}       & \begin{tabular}[c]{@{}l@{}}Ads, sponsored\\ stories or news\end{tabular}           & Ads                                                                                    \\ \hline
\end{tabular}
\caption{Some major online social media networks compared}
\label{tab:social_media_sites}
\end{table}

\section{Digital Media}
%Discussion of other media such as news sites, memes, pictures, and more
In addition to online social media sites, several other important sources of modern information spread mediums can be lumped together as ``digital media". Digital media in the context of information spread can include any online media source that spreads throughout a population. Often times, digital media takes the form of news sites and blogs, internet memes, pictures that become viral, and more. 

\subsection{News Sites and Blogs}
News sites are very common and simple to conceptualize. Generally speaking, many trusted news sites are simply online counterparts of a news source that has a newspaper or television presence. Sometimes, online-only news sources exist as well, particularly for niche topics such as news that highlights the latest tech gadgets, or an online news site that collects and distributes upcoming movie spoilers from inside sources. Most of the time, to be considered a legitimate news source, the organization and writers must be trusted to be reasonably impartial and put a reasonable effort into fact checking news before it is published online. Many sites try to pose as legitimate news, but in reality have a hidden agenda or interest which causes it to report unverified or even false stories and paint them as true. This practice is what is normally referred to as ``fake news".

Blogs, short for ``web logs", are essentially internet-based informal opinions and discussions on a specific topic. Their main purpose is to express an opinion or viewpoint and oftentimes encourage discussion in a comments section following the blog article. It is important to note that blogs do not attempt to present themselves as official news, however well written and thoughtful they may be. Still, they are an important medium for spreading information and especially opinions within a population. Especially popular blogs might become linked to an online social media site and spread far beyond their initial intended audience. While most blogs are text-based, they can also be comprised primarily of video, artwork, photographs, or audio (in the case of podcasts). 

\subsection{Internet Memes}
An internet meme is typically, but not necessarily a comedic piece of media (such as a video, image, hashtag, phrase, etc.) that spreads from one individual to another via the internet. Memes are often cultural symbols and social ideas that have a tendency to spread in a viral manner. One enduring example of an internet meme is the ``Rickrolling" prank in which people send a seemingly legitimate internet link which leads instead to a Rick Astley music video from the 1980s. Another good example is the popular trend of using photo editing software to alter movie posters in extreme and comical ways to express a point and post the new image on social media sites. Many memes that become widespread evolve over time to suit new situations and desired subjects of commentary.

Internet memes are an excellent way to examine information spread because they leave behind a digital footprint of where they have been on social media and the internet at large. While some may die out and others become viral, they are reasonably easy to trace compared to other types of information. Additionally, because they are usually satire and not directly opposed as some contentious information (but often merely ignored), internet meme spread study simplifies many of the more difficult and unpredictable elements of the study of information propagation. %Social Networks and Digital Media
\chapter{Literature Review}% Chapter titles are capitalized
Due to the breadth of areas of study required to have a fundamental understanding of mathematical modeling and control of online social media information spread, a review of previous works is divided into three main sections: social networks, mathematical modeling, and control. The social networks review will focus on important contributions to network theory, especially as it applies to online social networks. A review of mathematical modeling will discuss the major historical and modern methods of modeling spread interactions, especially in reference to information. Finally, the control review will examine some attempts at controlling information spread and discuss the benefits and shortcomings of the methodologies utilized.

\section{Social Theory and Social Networks}
While the early concepts of social theory and interactions have been touched upon decades before by Homans and others, Emerson provided an excellent summary and consolidation of priors works in psychology and sociology \cite{homans1958social}. He presented a social exchange theory to examine the exchanges between individuals and small groups, essentially stating that each individual behaves in accordance with a cost-benefit assessment of potential social interactions \cite{emerson1976social}. The overarching concepts, while not intended for modern online social media interactions, perhaps still hold true.

Pan and Crotts extend the concepts present in social exchange theory to online social media interactions, where they compare those who generate online information to those who consume, share, and comment on it, noting that online users are typically watchers and not producers, but still act in accordance with a risk versus reward mentality of sharing, producing, or commenting on content \cite{sigala2012social}. 

As the mediums by which information is being spread evolve, so too must the theories that attempt to describe and examine person-to-person communication. McLahan, a Canadian philospher and educator, contends that the ``media is the message". That is, the medium by which information is communicated within society has a much greater influence on a group than the specific content of the message being communicated \cite{mcluhan1995media}. With a decrease in the importance of traditional newspaper, television, and word-of-mouth word spread, internet news, blogs, and social media have emerged as a major vehicle for news spread. At first glance, it is easy to agree with McLuhan. The advent of the internet has changed our lives and how we get information, but has had little effect on the information itself in the majority of circumstances. 

How information propagates has become an increasingly important topic as communication and widely accessible digital news and media overtakes traditional news sources. Often times, this propagation, especially between different individuals or groups is known as ``information diffusion". Many researchers are focusing on the topological aspects of networks to study this phenomena. For example, Bakshy, Marlow, Rosenn, and Adamic examine the role of online information diffusion with a large-scale field experiment on sharing information with friends. By examining the role of both strong and weak ties within a large network, they determined that while strongly tied individuals are more influential within a group, weakly tied individuals are responsible for information diffusion between groups or networks \cite{bakshy2012role}. 

Other research focuses on the content of the information message as a major factor of its ability to spread. In Rappoport and Tsur's paper, an algorithm is developed and evaluated through use of Twitter hashtag extraction to demonstrate that the content itself has a significant impact on its ability to spread \cite{tsur2012s}.

In Weng's consolidated analysis, she asserts that information spread in online social networks have four main components: the actors that see and spread the message, the content of the message itself, the network topology through which the message must spread, and finally, the processes by which the message diffuses throughout the network topology. Weng concluded that very diverse topics can predict the popularity of a message within a community, while low diversity of messages tend to increase the influence of a single individual spreader. Finally, the network community's viewpoint influences information through social reinforcement and homophily to ``trap" information in and out of community nodes. As a result, early-stage information does not diffuse as an infectious disease as most other models assume \cite{weng2014information}.

Using an empirical study of news spread on social media networks such as Digg and Twitter, Lerman and Gosh extract data from the social network sites to demonstrate the critical role they play in information spread and how the network structure affects the information flow \cite{lerman2010information}. No explicit dynamics or modeling is given, but they demonstrate the importance and value of the empirical analysis of social media sites to support qualitative social theory with respect to online communities. 

\section{Mathematical Modeling}
There are two main methods of modeling information propagation: through network analysis and through maco-modeling of total population dynamics. This paper focuses on the later. The majority of this macro-modeling originates from mathematical epidemiology principles.

Kermack and McKendrick developed the first modern set of papers describing the transmission of communicable diseases as well as the concept of dividing individuals in a population into susceptible, infectious, and recovered classes \cite{kermack1932contributions}. Several variations of the Kermack-McKendrick model are used to describe epidemiological spreading, in which the susceptible-infected-recovered relationships are manipulated or reduced to suit the particular problem or disease. 

While diseases in many ways are similar to information as to how they are spread, the traditional Mermack-McKendrick model and the models it has directly influenced do not capture several elements that must be considered when information is the ``infection". In response to this need, Daley and Kendall developed a similar model using the same susceptible-infected-recovered classes to represent the flow of information via pair-wise contacts between individuals when spreading rumors within a group \cite{daley1965stochastic}. 

As a popular variant to the Daley-Kendall model, the Maki-Thomson model was developed to account for social factors using directed contacts of spreaders as well as postulating that initiating spreaders can become stiflers of a rumor if contacting other stiflers \cite{maki1973mathematical}. Again, the Maki-Thomson rumor spread model is insufficient to describe modern online social media interactions, but it does provide a solid foundation from which to further examine epidemiological models and apply them toward the study of online social media information spread. Several researchers have utilized the Maki-Thomson model as a basis for the macro-modeling of social information spread dynamics using variations similar to those of traditional disease epidemiology research.

Aside from disease epidemiology models, social communication and exchange have been studied in detail through marketing and advertising models. Specifically, the Vidale-Wolfe model and the Sethi model. Vidale and Wolfe modeled a continuous-time advertising model to link brand sales, market size, and brand sale decay \cite{vidale1957operations}. For certain online social information spread applications, the Vidale-Wolfe model appears promising for the modeling of modern online social network information spread, but does not account for a sustainable idea being spread once active advertisement ceases. The Sethi model expands upon the Vidale-Wolfe model by presenting advertising dynamics as a form of stochastic differential equation and adds a diffusion coefficient, as is consistent with several results of studies from social network structure examination \cite{sethi1973optimal}.

While none of these models precisely capture the needs of online social media information spread, especially when information may be contentious, they provide valuable starting points from which to develop new models. 

\section{Control}
Generally speaking, theoretical information spread control is performed via traditional engineering methods using the model dynamics discussed earlier. Popular topics often include marketing and political campaign information. For example, Kandhway and Kuri model rumor spread in a homogeneously mixed population using the Maki-Thomson rumor spread model. An optimal control problem is formed using an isoperimetric budget constraint (as a maximum advertising budget) on a campaigner with the goal of maximizing the spread of an information message \cite{kandhway2014optimal}. Unfortunately, the specific application of this is left fairly general, which could greatly influence the best model choice to utilize. Here, the assumption seems to be that the more people are aware of your campaign, the more success it will see. This assumption seems suspect, however, since it does not account for controversial or factually false messages, which could alter the results of the optimal strategy significantly. That said, it is sound to assume that by minimizing the number of ignorant people, more individuals will be aware of your campaign message and buy into it. Additionally, the model does not account for the effect of active attempts to campaign against a message within the population.

Kandhway and Kuri examine another optimal control problem of maximizing the spread of social network epidemics using the susceptible-infected (SI) model toward the application of campaigning on a fixed budget. Their model is treated as a mean field model assuming a very large population. This model is chosen over the SIR and SIS models because individuals are not seen as recovering from or forgetting the message learned during the course of the campaign period \cite{kandhway2016campaigning}. The choice of an SI model over and SIR or SIS model makes sense for simple advertisements and messages from a campaign that wishes to be spread and diffuse between groups, but ignores individuals that might halt or reverse the information diffusion by either making spreading redundant (hence discouraging further spread) or actively working against the diffusion via counter messages. This could especially be an issue in the case of controversial messages or ones that are simply not true. The division of nodes into classes based on their degrees of inter-connectedness presents a good way to exercise targeted control in a mixed population where resources are limited and must be spent to greatest effect. Additionally, neither control attempt takes special care in accounting for the online social media information spread proliferation. 

Altshuler, Shmueli, Zyskind, Lederman, Oliver, and Pentland proposed to execute an optimal campaign strategy under limited resources that automatically determines the number of interacting units and their class type in order to optimally allocate costs associated with them in network regions for the best campaign performance result \cite{altshuler2014campaign}. The role of computational social science is emphasized as the best way to model social systems such as campaign message interactions. A Gross Rating Points (GRP) metric is used, along with an exposure distribution to describe the campaign message spread quantitatively. With great progress made in the fields of computational social science, artificial intelligence, and mass data acquisition, it seems important going forward in any type of social information spread study to include such tools. If nothing else, it gives one of the best ways to actually collect data for online information exchange. With such massive amounts of data collected, heavy computer analysis and even basic artificial intelligence become requirements to successful data collection and processing. One major benefit of this control model is that it falls in line with the social theory that supports the importance of critical nodes and groups for information diffusion, such as Lerman and Ghosh \cite{lerman2010information}. %Literature Review
\chapter{Social Network Preliminaries} \label{chapter_preliminaries}
In this chapter, we will review some basic social network theory fundamentals. Understanding the essential theories, concepts, and terminology is critical in further discussing the topic of information modeling and control within online social networks. 

\section{Homophily and Filter Bubbles}
Meaning ``love of the same", homophily is a term coined in the context of social theory by Lazarsfeld and Merton in 1954 \cite{lazarsfeld1954friendship}. Essentially, it expresses the concept that similar individuals (or groups of individuals) tend to be drawn together within networks, with closer similarities resulting in closer network bonds. Likewise, people or groups with dissimilarity will tend toward involvement in completely separate social networks. Additionally, these network groups often induce positive feedback into themselves due to similarities between members, making the group ``likeness" bonds increasingly stronger \cite{kadushin2012understanding}. 

In a modern popular social media context, this phenomenon is often known as a filter ``bubble" \cite{bozdag2013bias}. Similar to the concept of a social ``echo chamber", a social filter bubble is an isolation of individual thought, perceptions, and news from opposing viewpoints due to their current belief systems, social media circles, and internet search tendencies. Evolving non-transparent technology has made filter bubbles increasingly effective, as personalized news streams, ads, and searches begin to dominate typical internet activity. Growing concern has arisen as to whether or not this trend is harming democratic ideals as these concepts enter public consciousness \cite{difranzo2017filter} following the recent, social media internet attributed, 2016 U.S. election results. Additionally, the knowledge of the existence of fake news and filter bubbles has eroded some public trust in traditional television, newspaper, and internet journalism. The following graph exemplifies an increasing trend of popular distrust of mass media over the years in the United States:

\begin{figure}[!htbp] \centering
  \includegraphics[width=0.7\linewidth]{figures/MassMediaTrust.eps}
  \caption{U.S. trust in mass media trends \cite{swift:2016}}
  \label{fig:MassMediaTrust}
\end{figure}

\section{Dyadic Relationships and Reciprocity}
In the realm of sociology, the simplest grouping consists of two individuals, also known as a dyad. An example of this might be a teacher and a student. Both have a connection to one another, as they interact with and influence the other within a small two-person network. In the case of the teacher-student example, there exists ``reciprocity", between the two individuals for the aforementioned reasons. In personal interactions, dyadic reciprocity is common, but once online social media interactions enter the picture, it is not hard to imagine several typical situations in which non-reciprocal relationships dominate. For example, consider an internet blogger with several hundred followers. The blogger may follow and reply to some of his or her readers, but for the most part, the blogger is not interacting with the readers and the relationship is purely one-sided. Returning to the teacher-student dyadic relationship, if the teacher simply lectures material and the student does not actively participate in course discussion (if any), then there exists no reciprocity in the relationship. In networking terms, these relationships can be called ``directed", as there is a one-sided, non-mutual connection between the individuals.

\section{Triads and Balanced Relationships}
Let us expand the simple person-to-person relationship to three individuals, each existing within the same network. With the addition of a third person, network analysis can truly begin because a society (however small) has emerged {\cite{simmel1950sociology}. As a result of the introduction of the third individual of the ``triad", the complexity of the relationships has greatly multiplied. Consider three individuals: persons A, B, and C. Person A is a good friend with person B, reciprocally. Person C is friends with person B, but does not know person A, however, person C follows the blog posts of person A due to common interest, but is not reciprocally followed. 

\begin{figure}[!htbp] \centering
  \includegraphics[width=0.7\linewidth]{figures/TriadRelationships.eps}
  \caption{Example of a simple triad relationship}
  \label{fig:TriadRelationships}
\end{figure}

Clearly, as the network grows in size by even a small amount, it becomes more complex due to the reciprocity of relationships, the presence or lack of intermediaries, and the number of individuals within the network group. To add further complexity, the concept of balance can be considered. Heider formalizes balance within a triad network as: ``In the case of three entities, a balanced state exists if all three relationships are positive in all respects, or if two are negative and one is positive" \cite{heider1946attitudes}. Heider contends that groups naturally tend toward this balanced triad state. As an example, consider a triad where two individuals dislike the third triad member. It is likely then, that the two disliking members will like one another, perhaps due to shared ideologies or opinions that cause them both to dislike the third person of the network. Eventually, the third person may even become isolated from the group or network entirely \cite{kadushin2012understanding}.

\section{Social Network Structures}
Let us consider Figure \ref{fig:SampleNetworkStructure}, a randomly generated network of fifty nodes, each representing an individual in a social network. These network sociograms were created with the widely used Social Network Visualizer software by Dimitris Kalamaras. A snapshot of the tool and some of its capabilities are shown in Figures \ref{fig:SNV_UI} and \ref{fig:SNV_Reports}.

\begin{figure}[!htbp] \centering
  \includegraphics[width=0.7\linewidth]{figures/SampleNetworkStructure.eps}
  \caption{Randomly generated fifty-node network}
  \label{fig:SampleNetworkStructure}
\end{figure}

\begin{figure}[!htbp] \centering
  \includegraphics[width=0.7\linewidth]{figures/SNV_UI.eps}
  \caption{User interface for the Social Network Visualizer tool}
  \label{fig:SNV_UI}
\end{figure}

\begin{figure}[!htbp] \centering
\begin{subfigure}[b]{0.4\textwidth}
  \includegraphics[width=0.9\linewidth]{figures/Centrality_Report.eps}
  \caption{Centrality}
\end{subfigure}
\begin{subfigure}[b]{0.4\textwidth}
  \includegraphics[width=0.9\linewidth]{figures/Clustering_Report.eps}
  \caption{Clustering}
\end{subfigure}
  \caption{Social Network Visualizer degree centrality and clustering reports}
  \label{fig:SNV_Reports}
\end{figure}
Note that the nodes in the network vary significantly in relation to one another. Some nodes are sparsely connected, others are very dense and connected to several neighbors. Additionally, clusters of nodes in close connection and proximity are also visually apparent. In order to understand online social networks, discuss them, and analyze them,  several commonly used concepts and terms must first be understood, beginning with individual distribution and position relative to other members of the network. 

\subsection{Density and Structural Holes}
Network density is defined as the number of direct actual connections divided by the number of possible direct connections in a network. A potential connection is a connection that could potentially exist between any two nodes, although it may not actually be connected. An actual connection is one that actually exists. \cite{lawyer2015understanding}. Equation \ref{eqn:Network_Density} gives the mathematical calculation for network density, where $n$ is the number of nodes in the network. Figure \ref{fig:Density} visually compares a sample sparse and dense network. 
\begin{equation}\label{eqn:Network_Density}
\left.\begin{aligned}
Potential \  Connections = \frac{n(n-1)}{2}\\
Network \  Density = \frac{Potential \  Connections}{Actual \  Connections}
\end{aligned}\right.
\end{equation}\\
\begin{figure}[!htbp] \centering
\begin{subfigure}[b]{0.4\textwidth}
  \includegraphics[width=0.9\linewidth]{figures/Low_Density.eps}
  \caption{Sparce network}
\end{subfigure}
\begin{subfigure}[b]{0.4\textwidth}
  \includegraphics[width=0.9\linewidth]{figures/High_Density.eps}
  \caption{Dense network}
\end{subfigure}
  \caption{Comparison of sparse and dense networks}
  \label{fig:Density}
\end{figure}
A real-life group such as a class or club would typically be fairly dense because each individual is usually acquainted with (or directly connected to) their fellow classmates or group members. Similarly, online groups with high levels of direct communication such as family social media groups or online game ``guilds" of sufficiently small size will be relatively dense. Higher levels of density often come paired with an increase of information spread and a sense of community along with the resultant inter-group social support structures. By their nature, small networks tend to be denser than large social networks. It's easy to know everyone in a class of twenty individuals, but knowing everyone in an entire school becomes increasingly unfeasible.

In direct contrast to the concept of density is what Burt refers to as ``structural holes" \cite{burt2009structural}. Imagine two dense networks comprised of individuals that mostly know one another and a single individual is a part of both groups, being their only common connection. If we imagine the networks combined into a single, larger grouping, there exists a structural hole within the new network, centered around the cluster-bridging individual. Figure \ref{fig:StructuralHoles} clearly illustrates a single connecting individual bridging two clusters within the same social network. One may naturally wonder why these groups are in the same network and not divided, but several real-world examples of structural holes in social networks are common. Politically different groups within the same country, rival teams within a sports league, and college courses taught by a single professor at two different school within the same city are all good examples of social network structural holes.

\begin{figure}[!htbp] \centering
  \includegraphics[width=0.7\linewidth]{figures/StructuralHoles.eps}
  \caption{Illustration of a structural hole}
  \label{fig:StructuralHoles}
\end{figure}

\subsection{Weak and Strong Ties}
The concept of weak ties is closely related to that of a structural hole, in that weakly tied social networks are linked by a few bridging individuals, such that two or more distinct group clusters can be readily identified. Practically speaking, weak ties help prevent large networks from being completely fragmented by facilitating the spread of information between segments. Other factors, however, help define a tie strength, such as the length of time individuals are acquainted, level of interaction, and how close in friendship or acquaintance individuals subjectively feel toward one another \cite{kadushin2012understanding}. Especially in online social networks, weak ties can play a critical role in information diffusion. Strong ties can be seen in the opposite fashion. They facilitate reinforcement of group values and tend to feed the same ideas and culture back into the group.

\subsection{Centrality and Distance}
In simplest terms, centrality describes how connected a node is to the network \cite{sigala2012social}. A centralized node will be highly connected to several other important nodes and hence have easier access to a number of network members when compared to a low centrality node. As there are many ways to define the importance of a node based on its connectivity, there are multiple methods used to quantitatively define centrality. Popular centrality measurements including degree centrality, closeness centrality, betweenness centrality, eigenvector centrality, and Katz centrality, to name a few. In Figure \ref{fig:Centrality}, nodes of high centrality are readily apparently by their high level of connectivity and importance to the network structure. The removal of these nodes would change the structure of the network considerably, while outlying nodes with few connections would keep the basic structure of the network intact. 
\begin{figure}[!htbp] \centering
  \includegraphics[width=0.7\linewidth]{figures/Centrality.eps}
  \caption{A random network demonstrating centrality and distance}
  \label{fig:Centrality}
\end{figure}
Degree centrality can be though of as a node's risk of catching whatever information (in this context) is flowing through the network immediately and represented mathematically as follows:
\begin{equation}\label{eqn:Degree_Centrality}
\left.\begin{aligned}
C_D(v)=deg(v),
\end{aligned}\right.
\end{equation}\\
\noindent where $v$ is the node of interest. Additionally, degree centrality can be expanded to the entire network group to measure network centrality, or the degree to which the network is centralized is determined by: 
\begin{equation}\label{eqn:Network_Centrality}
\left.\begin{aligned}
C_D(N) = \frac{\sum_{j=1}^{|V|}(C_D(v^*)-C_D(v_i))}{H}
\end{aligned}\right.
\end{equation}\\
\noindent where $v^*$ is the highest degree node of network graph $G$. $H$ is defined as:
\begin{equation}\label{eqn:Network Centality_H}
\left.\begin{aligned}
H= \sum_{j=1}^{|Y|}(C_D(y^*)-C_D(y_i)),
\end{aligned}\right.
\end{equation}
\noindent with $y^*$ as the node with the highest degree centrality in the network $Y$ that maximizes $H$.
Mathematically, closeness centrality (the most intuitive measure) is calculated as:
\begin{equation}\label{eqn:Closeness_Centrality}
\left.\begin{aligned}
C(x) = \frac{N-1}{\sum_{y}d(y,x)},
\end{aligned}\right.
\end{equation}\\
\noindent which is the reciprocal of ``farness", where $N$ is the total number of nodes in the network and $d(y,x)$ is the distance between the $x$ and $y$ vertices \cite{bavelas1950communication}.

Related to centrality is the idea of ``distance" between nodes of a network. Also known as a geodesic distance, network structure distance is formally defined as the distance between any two nodes is the length of the shortest path via the edges or binary connections between nodes \cite{bouttier2003geodesic}. 
Typically, distance is calculated using breadth-first traversal \cite{pemmaraju2003computational} or Dijkstra's algorithm \cite{dijkstra1959note}.
Consider again Figure \ref{fig:Centrality} and note that the highly connected central nodes can reach nearly any other node in only a few steps, by following their connection lines. Direct neighbors will be reachable in only a single step, while outer individual nodes will take only a few steps steps. In contrast, less centralized nodes will need at minimum, several steps (perhaps passing through these centralized nodes) to reach other outlying network members. The concepts of centrality and distance becoming increasingly important when discussing large populations such as cities or political groups. In the context of online social media, these principles remain true. A entrepreneurial or political leader will be a centralized individual within a country when making an online post or announcement, just as they would be if their information were to spread in traditional media outlets such as television and newspapers. On social networking sites such as Facebook, ``friends of friends" are a greater network distance from an individual than their core friend group, so receiving and relating information becomes slower and more difficult.

\subsection{Small World Networks}
Consider a network in which there exists no overlap between the personal networks of each individual if taken as a series of simple nodes and their direct neighbors. In this scenario, each new individual added to a network brings in an entire group of new network members that they and they alone have acquaintance. It's easy to see that networks organized in this fashion can attain extensive reach by adding only a few members. However, such groupings are not common in typical real-world networks, particularly when discussing online social networks. Friends have common friends (or friends of friends) that all know each other from the same college or club. Coworkers know most others in the office and do not befriend in entirely isolated groups. There are usually several non-unique individuals that have relationships from overlapping sources. These types of networks are known as ``small world" networks \cite{watts1998collective}. Small world networks are perhaps the most commonly discussed and analyzed due to both their limited scope and realistic inter-connectivity.

\subsection{Clusters, Cohesion, and Polarization}
The idea of social network clusters, are closely linked to Charles Cooley's concept of primary groups:
\begin{quote}
By primary groups I mean those characterized by intimate face-to-face association and cooperation. They are primary in several senses, but chiefly in that they are fundamental in forming the social nature and ideals of the individual. The result of intimate association, psychologically, is a certain fusion of individualities in a common whole, so that one's very self, for many purposes at least, is the common life and purpose of the group. Perhaps the simplest way of describing this wholeness is by saying that it is a ``we"; it involves the sort of sympathy and mutual identification for which ``we" is the natural expression. One lives in the feeling fo the whole and finds the chief aims of his will in that feeling \cite{cooley1909social}.
\end{quote}

\noindent In many ways, clusters are similar to Cooley's primary groups, but they do not overlap. Under cluster categorizations, one cannot be a member of multiple clusters at once. Sometimes there exists hierarchies and organization by which members identify themselves, but oftentimes, in large social networks especially, formalized categorizations can get messy and blurred even if they technically exist \cite{kadushin2012understanding}. In Figure \ref{fig:Polarization}, there are three distinct politically oriented blogs: liberal, moderate, and conservative, forming three distinct clusters within the online social bloggers network.

\begin{figure}[!htbp] \centering
  \includegraphics[width=0.7\linewidth]{figures/Polarization.eps}
  \caption{Network of U.S. political blogs by Adamic and Glance (2004) \cite{adamic2005political}}
  \label{fig:Polarization}
\end{figure}

Cohesion is a measure of network group connectivity in social groups. It defines the minimum number of individuals that must be removed from the group to cause it to dissociate. Ideally, a highly cohesive group will be connected to several members within the same cluster in a network such that severing individuals from the group does not cause the cluster to break apart to any substantial degree. Cohesive primary groups within a larger social network are often casually referred to as ``cliques". The strength or cohesion of cliques can be measured by their ability to pull together as a group to resist disruptive forces directed toward the clustered network group \cite{yang2016social}. For example, if someone challenges the beliefs and norms of a cohesive cluster, it will join together to reinforce those beliefs and norms.

In modern social network commentary, cluster polarization is a hot topic. Figure \ref{fig:Polarization} exemplifies a highly polarized political community in which the vast majority of online social network members blog with strong ideological tendencies, usually in direct opposition to another strongly cohesive cluster. Most members are either firmly liberal or conservative with only a small section of the network acting as moderate bloggers. The concepts of homofily and filter bubbles discussed earlier come into play in scenarios where a network is polarized, as members surround themselves with information with which they already agree. Concerns have been expressed over the dangers of this trend, especially with the advent of online social media sites where members of a cohesive clique can easily fall into their own bubbles of personalized news feeds, search recommendations, and YouTube video programs \cite{nikolov2015measuring}. Modeling and attempts at controlling highly polarized groups will be addressed in later parts in detail.

\section{The Adjacency Matrix}
In the previous sections, some network relationships were examined on a high level, but there was no mention of how to represent those relationships mathematically. One way to describe a network and its interrelationships is the adjacency matrix. 

Let us again consider the three node sociogram from the previous chapter, as shown in Figure \ref{fig:3_node_sociogram2}. Notice that each node pair of individual relationships has two elements of interest: direction and the presence of a connection. An adjacency matrix can be formed from the simple social network structure to show the mathematical relationships between each pair of the networked group. In the sample adjacency matrix presented in Table \ref{tab:adj_matrix}, $0$ represents no connection between the paired groups and $1$ represents a connection. It should be noted that connection is directional, so while one person may be connected to an adjacent individual, that second individual may not have a connection to the initial person. 
\begin{figure}[!htbp] \centering
  \includegraphics[width=0.6\linewidth]{drawings/3_node_sociogram.eps}
  \caption{Revisiting the three-node relationship mapping of three individuals}
  \label{fig:3_node_sociogram2}
\end{figure}

\begin{table}[]
\centering
\begin{tabular}{|l|l|l|l|}
\hline
           & \textbf{1} & \textbf{2} & \textbf{3} \\ \hline
\textbf{1} & 0          & 1          & 1          \\ \hline
\textbf{2} & 1          & 0          & 0          \\ \hline
\textbf{3} & 1          & 1          & 0          \\ \hline
\end{tabular}
\caption{Adjacency matrix of a sample three-node network}
\label{tab:adj_matrix}
\end{table}

The tabular matrix can be rewritten as a standard matrix for the purposes of future mathematical manipulation.
 %Social Network Preliminaries
\part{INFORMATION SPREAD MODELING}
%\chapter{LITERATURE REVIEW OF INFORMATION SPREAD}% Chapter titles are capitalized

\section{Social Network Information Diffusion}
%Information Diffusion on Online Social Networks
Weng asserts that information spread in online social networks have four main components: the actors that see and spread the message, the content of the message itself, the network topology through which the message much spread, and finally, the processes by which the message diffuses throughout the network topology.
Actors, which are individuals in the social network, are said to have limited attention capacity for information that is cared about. With so much information and messages on the internet, only so much can be deemed important enough to read and spread. Tie strength between the message's origin and the actor, topical interest level, homophily (or ``love of the same") social factors, and social group reinforcement all play roles in whether or not the actor will consume and spread any given set of information. Weng finds that only very strong (close-friend, for example) and very weak actor-actor ties attract the most attention in information diffusion, the strong ones for building and maintaining social relationships and the weak ones for receiving new information.

The role of content includes the topic itself, the sentiment it expresses, the language used in its presentation, and the culture in which the topic is of interest. By mapping memes and other information as nodes or clusters, a ``topic space". New information can be learned, therefore, by finding communities within the information or meme's co-occurrence network. By analyzing these topic spaces, Weng concluded very diverse topics can predict the popularity of a message within a community, while low diversity of messages tend to increase the influence of a single individual spreader. Finally, strongly tied individuals such as close friends are shown to correlate to a diversity of topics and sharing of said topics, while memes and other universally popular posts do not show a spread difference among differently tied groups.

When examining network topology, Weng concluded that the network community's viewpoint influenced information through social reinforcement and homophily to ``trap" information in an out of the community nodes. As a result, early-stage information does not diffuse as an infectious disease as most other models assume, when that diffusion is between node cluster communities. Because social networks are always evolving as information is being spread within the network, diffusion is highly dependent upon the network topology. Maximum-Likelihood Estimation and examination of social-based links from reposted information is used to determine that the concept of triadic closure strongly affects the formation of social links and network evolution along with ``short-cut" links based on the information that is being disbursed through reposts. Additionally, Weng concludes that strategies for following others in information spread varies greatly and linkage behaviors should be categorized and classified based on their network structural characteristics as well as their behavior characteristics in order to perform an analysis. \cite{weng2014information}

 %Literature Review Section (Modeling)
\chapter{Deterministic Models} \label{ch:DETERMINISTIC}

\section{Epidemic and Information Spread Models: Overview and Conventions}

Many deterministic models focus on a system's mathematical dynamics. That is, it describes the time dependence of a point in relation to its position (though it need not be a physical position). Examples of this can include a simple system like a swinging pendulum or complex systems such as the traffic flow on a highway \cite{contreras2016observability}. In deterministic dynamical models, only one future state can follow from the current state. Simple deterministic models of information spread have their mathematical dynamics origins as applications of epidemiology disease spread models. As such, much of the epidemiology terminology requires modification for ease of use and understanding in the context of information spread. The essential variables and parameters used here are outlined in Table \ref{tab:essential_terms}. 
Additionally, note that some of the common epidemic model acronyms are replaced in this text to properly reference them in the context of information spread applications. Specifically, epidemic spreaders, infected, and removed classes have been renamed as ignorants, spreaders, and recovered, respectively. 

\begin{table}[!htbp] \centering
\centering
\begin{tabular}{ll}
\textbf{Term}      & \textbf{Meaning}                                   \\
$I$                & Ignorants: Class which does not know the information   \\
$S$                & Spreaders: Class which is spreading the information \\
$R$                & Recovered: Class which no longer spreads the information \\
$\beta$            & Information spreading rate   \\
$\gamma$           & Information recovery rate   \\
$k$                & Average connectedness of individuals \\
\end{tabular}
\caption{Essential Variables and Parameters}
\label{tab:essential_terms}
\end{table}
\noindent In the simple deterministic models presented, a number of assumptions are made as follows:\\
(i)   Information is transmitted via spreader contact with an ignorant individual.\\
(ii)  Information is transmitted instantaneously upon contact.\\
(iii) All ignorant and spreading individuals are equally susceptible and transmittable, respectively.\\
(iv)  The population is fixed in size.\\
Several models and proposed models are introduced in this chapter. The model, its application uses, and some examples of application are presented. %in Table %\ref{}.

\section{The Ignorant-Spreader Model (IS)}
Consider the most simple case of information spread in which everyone in the population is either ignorant of the information or has heard of it and spreads it around to other members of the population. The Ignorant-Spreader model flow diagram is shown in Figure \ref{fig:IS_model}. Intuitively, applications of this model may not make sense at first glance, as one would imagine people to eventually lose interest in spreading a particular piece of information or new story. While this is typically true eventually, over a specified time-span, information can spread without stop. Good examples of this type of spreading would be long-term militant social or cultural movements or important news that everyone would care about over a short span of time like a local natural disaster alert. For the purposes here, one can think of the Ignorant-Spreader model as a subset of the Ignorant-Spreader-Recovered model discussed later, but with limitations. 
In the IS model, ignorants are always decreasing while spreaders are increasing as the information spreads throughout the population. 

\begin{figure}[!htbp] \centering
  \includegraphics[width=0.5\linewidth]{figures/IS_model.eps}
  \caption{Flow diagram for the Ignorant-Spreader model}
  \label{fig:IS_model}
\end{figure}

\noindent The system dynamics of the IS model are taken to be:
\begin{equation}\label{eqn:IS_dynamics}
\left.\begin{aligned}
\dot{i}(t) = -\beta k i(t)s(t)\\
\dot{s}(t) = \beta k i(t)s(t).
\end{aligned}\right.
\end{equation}

\noindent Plotting the system dynamics under typical parameters shows the following results:

\begin{figure}[!htbp] \centering
  \includegraphics[width=0.7\linewidth]{figures/IS_example.eps}
  \caption{Time Evolution of IS Model}
  \label{fig:IS}
\end{figure}

From Figure \ref{fig:IS} it is obvious that the spreaders eventually overtake the entire population at which point everyone is aware of the information and actively spreading it.

\section{The Ignorant-Spreader-Ignorant Model (ISI)}
Building upon the Ignorant-Spreader model, consider now the case in which ignorants become spreaders as they learn information, as before, but then recover from the information spreading state (due to disinterest over ``old news", for example) and return to a state at which the individual can became re-exposed to the information in order to again gain interest and spread it. Typical examples of this include the latest celebrity gossip or renewed interest in a technology product due to a new design. The flow diagram for the ISI model is presented in Figure \ref{fig:ISI_model}.

\begin{figure}[!htbp] \centering
  \includegraphics[width=0.5\linewidth]{figures/ISI_model.eps}
  \caption{Flow diagram for the Ignorant-Spreader-Ignorant model}
  \label{fig:ISI_model}
\end{figure}

\noindent The system dynamics of the ISI model are taken to be:
\begin{equation}\label{eqn:ISI_dynamics}
\left.\begin{aligned}
\dot{i}(t) = -\beta i(t) s(t)+\gamma s(t)\\
\dot{s}(t) = \beta i(t) s(t)-\gamma s(t).
\end{aligned}\right.
\end{equation}

\noindent Note that the main difference between the ISI model and the previous IS model dynamics is the addition of a stifling parameter $\gamma$ which represents spread information stagnating, causing the spreaders to return to the ingnorant class for a particular topic or topic set.

%\section{Special Case: The Ignorant-Spreader-Ignorant Model With Carriers}

\section{The Ignorant-Spreader-Recovered Model (ISR Maki-Thomson)}
Often times, information that is learned and spread is eventually simply forgotten as interest in the subject fades and people are not re-exposed to the same information. The flow diagram for the ISI model is shown in Figure \ref{fig:ISR_model}. Once a former spreader becomes recovered they are bored with spreading the information, disinterested in it, or is perhaps even aware of its true-false value and feels no need to further spread it. While in this state, individuals continue to interact pair-wise with other members of the network. Recovered individuals interacting with spreaders can sometimes signal to the spreader that the news being spread in that interaction is ``old news" and not worth spreading, hence converting the spreader to the recovered class. Likewise, two spreaders interacting can indicate that the information is already known and widespread and no longer warrants spreading, turning one of the pair into a Recovered state. Naturally, the chance of these state changes happening with any given pair are probabilistic and based on the stifling rate of the news. Examples of ISR types of information spread include factual news such as an earthquake occurring or even wide-spread universal reactions to a popular television program episode. It should be noted that ISI types of information spread can be viewed in the ISR form if one takes each new piece of information about a topic independently and specifically. While the screen size of the latest cellphone product might be modeled as an ISR system with its information propagation, the product brand as a whole could follow the ISI model as interest is renewed the the latest brand iteration. 

\begin{figure}[!htbp] \centering
  \includegraphics[width=0.5\linewidth]{figures/ISR_model.eps}
  \caption{Flow diagram for the Ignorant-Spreader-Recovered model}
  \label{fig:ISR_model}
\end{figure}

\begin{table}[!htbp] \centering
\centering

\begin{tabular}{ll}
\textbf{Interaction}      & \textbf{Result}                                   \\
$I + S + R = 1$           & Conservation of individuals in the population     \\
$I + S \rightarrow 2S$    & Spreader will infect an ignorant with the message \\
$S + S \rightarrow S + R$ & One spreader will recover if two interact         \\
$S + R \rightarrow 2R$    & Spreader will recover if contacting a recovered   \\
\end{tabular}
\caption{ISR Class Interactions}
\label{tab:sir_interactions}
\end{table}

\noindent The system dynamics of the ISR model are taken to be:
\begin{equation}\label{eqn:ISR_dynamics}
\left.\begin{aligned}
\dot{i}(t) = -\beta k i(t)s(t)\\
\dot{s}(t) = \beta k i(t)s(t) - \gamma k s(t)[s(t)+r(t)]\\
\dot{r}(t) = \gamma k s(t)[s(t)+r(t)].
\end{aligned}\right.
\end{equation}

\begin{figure}[!htbp] \centering
  \includegraphics[width=0.7\linewidth]{figures/ISR_example.eps}
  \caption{Sample Time Evolution of the ISR Model}
  \label{fig:ISR}
\end{figure}

\section{The Basic Reproductive Number: Spreading or Forgotten?}

One useful metric for examining whether or not information will spread through a population or dissipate before any significant number of people become aware of it is the basic reproductive number, denoted by $R_0$. It was first used in its modern form in 1952 by George MacDonald to model the spread of malaria throughout a population, but can easily be applied and envisioned in a variety of systems in which a growth and decay factor are at odds with one another. To imagine it simply, if information is being spread with more ``force" than it is being quelled or forgotten, then the information takes on epidemic properties and spreads throughout the population, otherwise the information will die out in the long run. $R_0$ is a threshold condition.
Formally, $R_0$ is defined as the expected number of secondary cases that arise from a single spreader in a population that is otherwise ignorant of the information. $R_0$ is not a rate, but instead dimensionless. 
\noindent In general, the basic reproductive number for information spread is defined as:\\
\begin{equation}\label{eqn:reproductive_number}
\left.\begin{aligned}
R_0 = \frac{\gamma + \beta}{\gamma} > 1 \\
\frac{\beta }{\gamma } > 0
\end{aligned}\right.
\end{equation}

\noindent The value of the basic reproduction number can have a large effect on the extent to which news and information spreads, if at all. Note in Figure \ref{fig:IRC_R0_low} the reproductive number of the information does not allow it to reach the entire Ignorant class, but in Figure \ref{fig:IRC_R0_high} the information spreads to most of the population quickly followed by a fast recovery. 

\begin{figure}[!htbp] \centering
  \includegraphics[width=0.7\linewidth]{figures/ISR_R0_low.eps}
  \caption{$R_0$ Comparison: Low $R_0$ value}
  \label{fig:IRC_R0_low}
\end{figure}

\begin{figure}[!htbp] \centering
  \includegraphics[width=0.7\linewidth]{figures/ISR_R0_high.eps}
  \caption{$R_0$ Comparison: High $R_0$ value}
  \label{fig:IRC_R0_high}
\end{figure}

\section{Proposed Ignorant-Spreader-Recovered Model in Social Media}

While the ISR information spread model works well for ``traditional" forms of pairwise interactions, special considerations must be made when it concerns modern social media networks. What makes digital social media so special? Because interactions occur very rapidly and deliberately. When one is immediately notified of an inter-network social media posting and opts to re-post or otherwise spread the information, the spreading and stifling parameters can swing wildly based on cultural whims. Additionally, when an individual in a social network has recovered a specific news or information item, that individual is not putting themselves in a position to actively have a pairwise contact the spreader class (at least concerning that news). This results in the spreader class being unchecked from recovered class members and hence spreaders only realizing that the news they spread is widely known upon contact with other spreaders. 

\noindent The resulting equations for an ISR system in social media are as follows:
\begin{equation}\label{eqn:ISR_dynamics_sm}
\left.\begin{aligned}
\dot{i}(t) = -\beta k i(t)s(t)\\
\dot{s}(t) = \beta k i(t)s(t) - \gamma k s(t)[s(t)]\\
\dot{r}(t) = \gamma k s(t)[s(t)].
\end{aligned}\right.
\end{equation}

\begin{figure}[!htbp] \centering
  \includegraphics[width=0.7\linewidth]{figures/ISR_sm.eps}
  \caption{Sample time evolution of the ISR model for social media}
  \label{fig:ISR_sm}
\end{figure}

\noindent It is noteworthy to observe that typically in digital social media systems, information spreads more rapidly and decays at a slower rate than traditional non-digital network rates, partially due to the lack of recovered class stifling factors.

\section{Proposed Ignorant-Spreader-Recovered Model in Social Media with Decay}

Consider the situation in which digital social media information or news is drawn out over a long period of time. Information will be learned, spread, and recovered from as before, but a natural human disinterest of dated news becomes a significant factor. To model this behavior, a simple exponential forgetting factor is added to the spreader dynamics, similar to the Vidale-Wolfe advertising model \cite{vidale1957operations}. The decay is balanced by the dynamics of the recovered class, as individuals who stop caring about a topic over time effectively act as recovered.

\noindent The dynamics equations for a social media ISR system with decay are as follows:
\begin{equation}\label{eqn:ISR_dynamics_decay}
\left.\begin{aligned}
\dot{i}(t) = -\beta k i(t)s(t)\\
\dot{s}(t) = \beta k i(t)s(t) - \gamma k s(t)[s(t)]-\delta s(t)\\
\dot{r}(t) = \gamma k s(t)[s(t)]+ \delta s(t).
\end{aligned}\right.
\end{equation}

\begin{figure}[!htbp] \centering
  \includegraphics[width=0.7\linewidth]{figures/ISR_d.eps}
  \caption{Sample time evolution of the ISR model for social media with decay}
  \label{fig:ISR_d}
  \end{figure}

\noindent Over long periods of digital media information spreading throughout a network, spreader decay reshapes spreader behavior to more closely resemble traditional ISR information spread curves. Obviously, the relevance and ``juiciness" of the news or information will greatly influence the rate of natural spreading decay.

\section{Proposed ISCR Model for Contentious Information Spread} 

In order to describe a single group where ``contentious information" is being distributed, a special model is required to account for not only spreaders, but also counter spreaders of the information. The traditional ISR model serves as a good base for the new model, but has its shortcomings when examining the spread of contentious information. Examples of when to apply this ISCR model include populations where ``fake news" is being spread, public panics have started over false information that must be quelled by the government, and the spreading of false celebrity death announcements on social media. At the core of this modified model is the idea of a ``counter spreader" which acts just like the spreader, but is pushing against the spreader information with an opposite message. As such, the pairwise interactions of an information based ISR model (such as the popular Maki-Thomson model) still apply in the ISCR model, but with a new class of individual present in the population as a possible state. Here, $I$ represents the ignorant class, which is unfamiliar with the information. $S$ represents the spreader class, which knows the information and is actively spreading it to each contacted class. $C$ represents the counter-spreader class, which knows the information and is actively spreading a counter-information campaign against the information being spread by the spreader class. Finally, $R$ represents the recovered class, which knew about the information at one time, possibly spread or counter-spread it, but has become uninterested in further spreading of the information due to simple disinterest, a belief that the information is already widely known, or because the class has determined that the information is proven false and should not be spread further. Inter-class interactions evolve as shown in Table \ref{tab:sirc_interactions}.

\begin{table}[!htbp] \centering
\centering

\begin{tabular}{ll}
\textbf{Interaction}    & \textbf{Result}                                   \\
$I + (S + C) + R = 1$     & Conservation of individuals in the population     \\
$I + S \rightarrow 2S$    & Spreader will infect an ignorant with the message \\
$I + C \rightarrow 2C$    & Counter will infect an ignorant with the message  \\
$S + S \rightarrow S + R$ & One spreader will recover if two interact         \\
$C + C \rightarrow C + R$ & One counter will recover if two interact          \\
$S + R \rightarrow 2R$    & Spreader will recover if contacting a recovered   \\
$C + R \rightarrow 2R$    & Counter will recover if contacting a recovered   
\end{tabular}
\caption{ISCR class interactions}
\label{tab:sirc_interactions}
\end{table}
\begin{figure}[!htbp] \centering
  \includegraphics[width=0.7\linewidth]{drawings/SIRC_Model.eps}
  \caption{Flow diagram for ISCR model class interactions.}
  \label{fig:sirc_flow}
\end{figure}

Figure \ref{fig:sirc_flow} shows a flow diagram for the class interactions, where parameters $\beta$, $\alpha$, $\gamma$, and $\mu$ represent spread rate, counter-spread rate, stifle rate, and counter stifle rate, respectively. The parameter $\omega$ represents the willingness of spreaders to listen to counter-spreader information ($\omega_1$) and vice-versa ($\omega_2$), which can take a positive or zero value in the case of completely hardline spreader and counter-spreader views. These parameters allow for spreaders and counter-spreaders to potentially examine the opposite viewpoint and ``change sides."

\noindent The system dynamics of the model can now be given as:
\begin{equation}\label{eqn:SIRC_dynamics}
\left.\begin{aligned}
\dot{i}(t) = -\beta k i(t)s(t) - \alpha k i(t)c(t)\\
\dot{s}(t) = \beta k i(t)s(t) - \omega_1 k s(t)c(t) + \omega_2 k s(t)c(t) - \gamma k s(t)[s(t)+r(t)]\\
\dot{c}(t) = \alpha k i(t)c(t) + \omega_1 k s(t)c(t) - \omega_2 k s(t)c(t) - \mu k c(t)[c(t)+r(t)]\\
\dot{r}(t) = \gamma k s(t)[s(t)+r(t)] + \mu k c(t)[c(t)+r(t)].
\end{aligned}\right.
\end{equation}
\noindent Plotting the system dynamics and varying the parameters leads to the results in the proceeding figures.

\begin{figure}[!htbp] \centering
  \includegraphics[width=0.7\linewidth]{figures/sirc_no_counters.eps}
  \caption{No counter-spreaders}
  \label{fig:sirc_no_counters}
\end{figure}

\begin{figure}[!htbp] \centering
  \includegraphics[width=0.7\linewidth]{figures/sirc_spread_dominant.eps}
  \caption{Spreaders dominate}
  \label{fig:sirc_spread_dominant}
\end{figure}

\begin{figure}[!htbp] \centering
  \includegraphics[width=0.7\linewidth]{figures/sirc_counter_dominant.eps}
  \caption{Counter-spreaders dominate}
  \label{fig:sirc_counter_dominant}
\end{figure}

\begin{figure}[!htbp] \centering
  \includegraphics[width=0.7\linewidth]{figures/sirc_even.eps}
  \caption{Even mix of spreaders and counter-spreaders}
  \label{fig:sirc_even}
\end{figure}

\begin{figure}[!htbp] \centering
  \includegraphics[width=0.7\linewidth]{figures/sirc_even_S_2x_receptive.eps}
  \caption{Spreader twice as receptive to outside influence as counter-spreader}
  \label{fig:sirc_even_S_2x_receptive}
\end{figure}

\begin{figure}[!htbp] \centering
  \includegraphics[width=0.7\linewidth]{figures/sirc_spreader_dominant_receptive.eps}
  \caption{Dominant spreader is 4.6 times as receptive to outside influence as counter-spreader}
  \label{fig:sirc_spreader_dominant_receptive}
\end{figure}

\begin{figure}[!htbp] \centering
  \includegraphics[width=0.7\linewidth]{figures/sirc_S_2x_stifle.eps}
  \caption{Spreader is stifled twice as strongly as counter-spreader}
  \label{fig:sirc_S_2x_stifle}
\end{figure}

\begin{figure}[!htbp] \centering
  \includegraphics[width=0.7\linewidth]{figures/sirc_C_2x_spread.eps}
  \caption{Counter-spreader is spreading twice as strongly as spreader}
  \label{fig:sirc_C_2x_spread}
\end{figure}

It is observable that spreading strength changes have the greatest effect, followed by stifling strength. Receptivity to an opposing view has a smaller, but still significant effect on information spread. Counter-spread information effectively overtakes (counters) the initial spreading view once the receptivity to outside views reaches approximately 4.6 times the receptivity of the counter-spreaders to agreeing with the initial information. 

It should be noted that counter-spreaders are naturally unlikely to be receptive to information that has been proven objectively false. As such, they will have a favorable spreader to counter-spreader conversion rate compared to the spreaders when there are no remaining members of the ignorant class to become spreaders of information that has not been researched or fully absorbed, as in the case of ``fake news" and similar contentious information scenarios.

\section{Proposed Hybrid ISCR Model} %\label{ch:SIRC_HYBRID} 
Often times, situations arise where two (or more) primary group ``communities" dominate in their view and spread of a certain type of information. In the real world, this is a very common situation. For example, in the United States, conservatives and liberals often form information communities that oppose one another on political information and interpretation. One country may be a mega-community and oppose a similar community from another country in how they are accepting and spreading news relating to inter-country relations. The ISCR model covers situations where potentially contentious information is being spread within a mostly homogeneous population. What it does not do well, however, is describe how information spreads or diffuses between two different, especially polarized groups. It is therefore necessary to modify the ISCR model to account for a hybrid case.

In the hybrid model, we assume two main polarized communities for simplicity, $ISCR_1$ and $ISCR_2$, which can be extended to any number of polarized groups. We assume there are some individuals within each community that has contact with the other. Perhaps they are moderates or friends who subscribe to a different political view or family members from a home country who are not part of one's current information community, but act as a link to another country-community. No matter the reason, some individuals act as diffusion elements of information between the two communities. The influence and connection strength of the overall collection of these individuals is denoted by a pair of directional constants $a_{12}$ and $a_{21}$, where $a_{12}$ is the strength of information flow from group $ISCR_1$ to  group $ISCR_2$ and $a_{21}$ is the strength of information flow from group $ISCR_2$ to  group $ISCR_1$, expressed as a percentage of cross-group receptivity. For highly polarized groups, these constants will be very small fractions of one and for completely isolated communities they will be equal to zero, reducing the hybrid model to the standard ISCR model discussed previously. Figure \ref{fig:ISCR__hybrid_model} shows the high-level flow diagram for the proposed hybrid ISCR model.
\begin{figure}[!htbp] \centering
  \includegraphics[width=0.7\linewidth]{figures/ISCR_hybrid_model.eps}
  \caption{Flow diagram for ISCR two-community interactions.}
  \label{fig:ISCR__hybrid_model}
\end{figure}

\noindent Modifying the previous ISCR model system dynamics for one group, we obtain:
\begin{equation}\label{eqn:SIRC1_dynamics}
\left.\begin{aligned}
\dot{i}_1(t) = -\beta_1 i_1(t)s_1(t) - \alpha_1 i_1(t)c_1(t) - a_{21} \beta_2 i_1(t)s_2(t) - a_{21} \alpha_2 i_1(t)c_2(t)\\
\dot{s}_1(t) = \beta_1 i_1(t)s_1(t) - \omega_{11} s_1(t)c_1(t) + \omega_{12} s_1(t)c_1(t) - \gamma_1 s_1(t) + a_{21} \beta_2 i_1(t)s_2(t)\\
\dot{c}_1(t) = \alpha_1 i_1(t)c_1(t) + \omega_{11} s_1(t)c_1(t) - \omega_{12} s_1(t)c_1(t) - \mu_1 c_1(t) + a_{21} \alpha_2 i_1(t)c_2(t)\\
\dot{r}_1(t) = \gamma_1 s_1(t) + \mu_1 c_1(t).
\end{aligned}\right.
\end{equation}
\noindent Likewise, for the second ISCR group the following dynamics are obtained:\begin{equation}\label{eqn:SIRC2_dynamics}
\left.\begin{aligned}
\dot{i}_2(t) = -\beta_2 i_2(t)s_2(t) - \alpha_2 i_2(t)c_2(t) - a_{12} \beta_1 i_2(t)s_1(t) - a_{12} \alpha_1 i_2(t)c_1(t)\\
\dot{s}_2(t) = \beta_2 i_2(t)s_2(t) - \omega_{21} s_2(t)c_2(t) + \omega_{22} s_2(t)c_2(t) - \gamma_2 s_2(t) + a_{12} \beta_1 i_2(t)s_1(t)\\
\dot{c}_2(t) = \alpha_2 i_2(t)c_2(t) + \omega_{21} s_2(t)c_2(t) - \omega_{22} s_2(t)c_2(t) - \mu_2 c_2(t) + a_{12} \alpha_1 i_2(t)c_1(t)\\
\dot{r}_2(t) = \gamma_2 s_2(t) + \mu_2 c_2(t).
\end{aligned}\right.
\end{equation}

Following the diffusion terms in the dynamics, the system can be loosely described as individuals from one group spreading their group's information and opinions on a topic to an opposing community via their friends or relatives. The receptiveness of one group to believe and accept the views of another will determine how much the extra-group spreader influences the opposing group.

Consider the two $ISCR$ groups shown in Figure \ref{fig:hybrid_sirc_initial}. The $ISCR_1$ community is predominantly influenced by the spreader information with minor counter-spreader information influence, while the opposite is true for the $ISCR_2$ community, being influenced mostly by the counter-spreader information (generally the information that contradicts or opposes the Spreader information). As an example, $ISCR_1$ spreaders might be mainly spreading an unverified rumor about an opposing political party candidate in the $ISCR_2$ community. Some members of $ISCR_1$ believe the initial information is a lie and are counter-spreaders in that community. The $ISCR_2$ community has their own contentious information concerning an opposing candidate from the $ISCR_1$ community with similar levels of dissent within their own group. Since the  $a_{12}$ and $a_{21}$ constants are zero, there is no diffusion with which to cause interaction between the communities. 

\begin{figure}[!htbp] \centering
  \includegraphics[width=0.7\linewidth]{figures/hybrid_sirc_initial.eps}
  \caption{Two Initial ISCR Groups: Spreader dominant and Counter-spreader dominant}
  \label{fig:hybrid_sirc_initial}
\end{figure}

With an equal amount of cross-interactions between the two communities (Figure \ref{fig:hybrid_sirc_equal}), the influence of counter-spreaders from $ISCR_2$ counteracts the influence of $ISCR_1$ spreaders and vice-versa. This, of course, assumes that in the polarized groups, the information of $ISCR_1$ spreaders is the same message as $ISCR_2$ counter-spreaders. This need not be the case, but serves to simplify the system for example purposes. 

\begin{figure}[!htbp] \centering
  \includegraphics[width=0.7\linewidth]{figures/hybrid_sirc_equal.eps}
  \caption{Two ISCR Groups: Equal Bidirectional Diffusion}
  \label{fig:hybrid_sirc_equal}
\end{figure}

When there is mismatched receptivity between the communities, dramatic shifts can occur in the way popular beliefs are altered via information spread. In Figure \ref{fig:hybrid_sirc_a21} the $ISCR_1$ community is more receptive to the $ISCR_2$ community and is hence more likely to listen to the messages of the diffuser spreaders and counter-spreaders. In this case, because the counter-spreaders are dominant in the $ISCR_2$ group (and many of these counter-spreaders are being heard), there is a large resurgence of counter-spreaders in $ISCR_1$, while keeping the spreaders of $ISCR_2$ relatively low (as they are not nearly as receptive to outside influence). 

\begin{figure}[!htbp] \centering
  \includegraphics[width=0.7\linewidth]{figures/hybrid_sirc_a21.eps}
  \caption{Two ISCR Groups: Skewed receptivity between groups}
  \label{fig:hybrid_sirc_a21}
\end{figure}

As skewed receptivity mismatching increases to where one group is about seven times as receptive as the other, we find a ``tip over point" where (all else being equal) the opposing community's information beliefs equal that of the primary community (Figure \ref{fig:hybrid_sirc_a21_cancel}). After this point, the information beliefs of the opposition group begin to dominate both groups (Figure \ref{fig:hybrid_sirc_a21_plus}). It should be noted that there are significant diminishing returns on the effect of receptivity to opposing polarized communities. 

\begin{figure}[!htbp] \centering
  \includegraphics[width=0.7\linewidth]{figures/hybrid_sirc_a21_cancel.eps}
  \caption{Two ISCR groups: receptivity tipping point}
  \label{fig:hybrid_sirc_a21_cancel}
\end{figure}

\begin{figure}[!htbp] \centering
  \includegraphics[width=0.7\linewidth]{figures/hybrid_sirc_a21_plus.eps}
  \caption{Two ISCR groups: dominant inter-community belief}
  \label{fig:hybrid_sirc_a21_plus}
\end{figure}

Clearly, between polarized groups, diffusion plays a significant role in changing the dominant information beliefs of a community. Generally speaking, if two groups are truly polarized, outside influences will be felt but the impact will be minor. A fairly strong amount of receptiveness to outside opposing communities would be required to have lasting change, and a situation in which one group is several times more receptive than the other is unlikely. Still, it is of value to objectively observe and model the influence groups have on one another along a spectrum of non-mutual receptivity levels.

\section{Proposed ISSRR Model for Contentious Information}
While the proposed ISCR model describes two primary groups in which opposing class members seek to turn the other to their side of belief of contentious information, it does not describe a situation in which there are multiple final recovered states of contentious information belief. A new model must be proposed to handle this type of social network contentious information spread.

Consider a basic ISR modeling of a polarized social network group that is presented with contentious information, leading to two final recovered beliefs, representing individuals who eventually make up their mind concerning the information and have lost interest in it, pending any relevant new information. As the new information is discovered by the network population, individuals form an initial opinion and many will choose to begin to spread the information opinion stance they have decided to follow among their friends and fellow social network members. At this point, two such spreader groups will arise, each pushing people via online social media tweets and posts to come to their ``side" of the contentious information in the hopes of gaining additional spreaders who, hopefully, finally settle on their preferred of the two recovered states. We shall call this the Ignorant-Spreader-Spreader-Recovered-Recovered (ISSRR) model, which is visually depicted in Figure \ref{fig:ISSRR_Model}.

\begin{figure}[!htbp] \centering
  \includegraphics[width=0.7\linewidth]{figures/ISSRR_Model.eps}
  \caption{Proposed ISSRR model class interaction overview}
  \label{fig:ISSRR_Model}
\end{figure}

Notice that in this model, after informed individuals split into their respective spreader groups, they will reciprocally interact with varying levels of balance with spreaders of the opposing viewpoint and eventually settle into a recovered state of one of the two contentious sides. Aside from the typical ignorant class becoming spreaders interactions from previous models, we must account for spreader-spreader interactions and the strength of tendency to achieve one of the two recovered states, based on that interaction. In other words, which side has the greater ability to compel the other, based on a demonstrable truth discovered to resolve the contention, assumed authority of the spreaders, and other factors when compared to the tendency of a group to stay with their original line of thinking will ultimately determine the relative outcome of the two competing sides of information distribution and belief. 

The practicality of an ISSRR model is evident in several real world situations. Social and cultural issues a community must agree upon with no clear objective right or wrong answer is one candidate for such a model. Political campaigning between candidates or measures that represent differences in ideologies, not necessarily value, can also be described with an ISSRR model. Even the concept of fake news mentioned in previous sections can be modeled under this model, provided the competing news stories are supported by highly polarized primary groups, where final-state recovered individuals settle on a belief in the actual or fake news.

The proposed ISSRR model takes the form of the following set of differential equations:
\begin{equation}\label{eqn:ISSRR_dynamics}
\left.\begin{aligned}
\dot{i}(t) = -\beta_1 i(t)s_1(t) - \beta_2 i(t)s_2(t)\\
\dot{s}_1(t) = \beta_1 i(t)s_1(t) + (d_{12}-d_{21})s_1(t)s_2(t) - (\gamma_{11}+ \gamma_{12})s_1^2(t)\\
\dot{s}_2(t) = \beta_2 i(t)s_2(t) + (d_{21}-d_{12})s_1(t)s_2(t) - (\gamma_{22}+ \gamma_{21})s_2^2(t)\\
\dot{r}_1(t) = \gamma_{11}s_1^2(t) + \gamma_{21} s_2^2(t)\\
\dot{r}_2(t) = \gamma_{22}s_2^2(t) + \gamma_{12} s_1^2(t),
\end{aligned}\right.
\end{equation}
\noindent where $d$ is the relative strength of influence and connection of one information-spreading group over the other. Note that this model assumes a lack of active online social media interactions on the part of the recovered class under the assumption that those who are bored, convinced of, or ``over" a topic will not be reaching out online to further engage the social network and at best will merely spectate or ``lurk" as spreaders continue to spread and argue positions. This departs from traditional rumor spread models.

Two sample scenarios were simulated using the presented ISSRR dynamics. In the first scenario, presented in Figure \ref{fig:ISSRR_1}, the first group has a greater ability to spread information via online social networks compared to the second group. Perhaps the first group is generally more internet savvy or culturally willing to share information over the internet, such as is often the case with younger versus older internet users. The two groups are equally likely to cease caring about the topic once they have made up their minds on a stance and are also both likely to favor their initial group's opinion trend over the opposing side. Finally, neither group has significant social network influence over the other. Note that the simple ability to spread more effectively than the side of an opposing opinion has a major influence over the eventual total group recovered opinion distribution, all else being equal.

\begin{figure}[!htbp] \centering
  \includegraphics[width=0.7\linewidth]{figures/ISSRR_1.eps}
  \caption{Proposed ISSRR: Scenario 1}
  \label{fig:ISSRR_1}
\end{figure}

In the second scenario, as shown in Figure \ref{fig:ISSRR_2}, both subgroups are equally effective at spreading their opinion and equally likely to grow disinterested in any given opinion, once they have made up their mind on the topic. Unlike in the first scenario, here the second group has greater influence over the first group. This quickly leads the dynamics to result in much greater numbers of individuals to ultimately settle on the second group's information-opinion.

\begin{figure}[!htbp] \centering
  \includegraphics[width=0.7\linewidth]{figures/ISSRR_2.eps}
  \caption{Proposed ISSRR: Scenario 2}
  \label{fig:ISSRR_2}
\end{figure} %Determinisitic Models
\chapter{Stochastic Models}

\section{Stochastic Models: What Are They and Why Are They Needed?}
In the previous section, deterministic models were presented and examined. These models assumed deterministic state values at each point in time to represent the system. While choosing a good assumed value for a state is very useful in simplifying a complex system, it should not be forgotten that real information spread systems (and most systems for that matter) are probabilistic in nature. In such systems, models and simulations must account for a range of variable values in the form of a chosen probability distribution. These models are known as stochastic models. 

While stochastic models greatly increase the complexity of a system model, both conceptually and mathematically, they also provide valuable insight into the bounds of a system's expected behavior. After all, it would be both unintuitive and unreasonable to expect an information spread model to fully predict (down to each individual) the depth and evolution of any given information element as it spreads throughout a population. Why? Because people themselves are mathematically probabilistic and unpredictable. As such, when many person-critical systems are modeled, a stochastic-minded approach to modeling is best. For example, proposed tax rates for vehicles based on miles traveled is ideally modeled using stochastic differential equations because everyone does not have the same driving habits or vehicles \cite{verma2016modeling}.

In this chapter, two stochastic information spread models are presented: a traditional information spread ISI model and an ISR information model, specifically tailored to social media information. Like other information spread models, these models are adapted from mathematical epidemiology models \cite{allen2008mathematical} with special consideration paid to the naming conventions of variables, a negligible birth and death rate over the short time it takes to spread information, and state transition differences due to human social media behavior. The same general principles presented here for the ISI and ISR models can be applied to other information spread models to achieve their coinciding stochastic model.

\section{Stochastic ISI Information Model}
Recall that in the Ignorant-Spreader-Ignorant model, the ignorant class learns and gains interest in a topic, spreads the information to others until said information becomes old or boring, and then returns to the ignorant state until the same topic or family of topics has new and interesting information available to learn to rekindle a desire to spread again. The stochastic ISI model is no different in concept, but must be handled differently due to the probabilistic nature of the system. 

Consider the expanded ISI deterministic system, where $N=I+S$ as the total population (for simplicity, $N$ can be set to 1 to simulate a population percentage as is done elsewhere in this paper). For ease of understanding, classes are generally displayed without time $t$ unless required, though they remain a function of time.

\noindent The system dynamics of the IS model are taken to be:
\begin{equation}\label{eqn:ISI_dynamics_2}
\left.\begin{aligned}
\dot{i} = -\frac{\beta}{N} i s+\gamma s\\
\dot{s} = \frac{\beta}{N} i s-\gamma s
\end{aligned}\right.
\end{equation}\\
\noindent Now, let us examine the following It\^o SDE:
\begin{equation}\label{eqn:Ito_ISI}
\left.\begin{aligned}
\frac{dS}{dt}= \mu(S) + \sigma(S) \frac{dW}{dt},
\end{aligned}\right.
\end{equation}
\noindent where $W$ is the Wiener process. If Euler's method is utilized, the system model converges to the It\^o SDE if specific growth and smoothness conditions are met, just as in stochastic epidemic models \cite{allen2008mathematical}. For this ISI model, $\mu(S)=b(S)-d(S)$ is the deterministic growth and decay of the information as before, and $\sigma(S)=\sqrt{b(S)+d(S)}$ is the stochastic element of the system, where
\begin{equation}\label{eqn:Ito_coeff}
\left.\begin{aligned}
b(S) = \frac{\beta}{N} S(N-S) \nonumber \\
d(S)= \gamma S. \nonumber
\end{aligned}\right.
\end{equation}
\noindent By expressing the dynamics in terms of the spreader class and making the above substitutions into the It\^o SDE, the stochastic ISI information spread model can be obtained in the form of the following SDE:
\begin{equation}\label{eqn:ISI_SDE}
\left.\begin{aligned}
\frac{dS}{dt} = \frac{\beta}{N} S(N-S)- \gamma S + \sqrt{\frac{\beta}{N} S(N-S)+\gamma S} \frac{dW}{dt}.
\end{aligned}\right.
\end{equation}
\noindent Plotting the stochastic ISI dynamics in Figure \ref{fig:ISI_SDE} with a set of stochastic realizations, we can see that the sample plots closely follow the deterministic ISI expected dynamics when taken as a whole.
\begin{figure}[!htbp] \centering
  \includegraphics[width=0.7\linewidth]{figures/ISI_SDE.eps}
  \caption{Stochastic realizations and the deterministic result of an ISI network}
  \label{fig:ISI_SDE}
\end{figure}

\section{Stochastic ISR Information Model and Social Media Spread}
Just as with the stochastic ISI model, similar assumptions and processes can be applied to the ISR information spread model, given a reasonably normal distribution for random variables. Two cases of the ISR information spread model are explored here: traditional and social media. 

In the traditional model, the ignorant class learns and hears about new information from traditional media and social sources such as word-of-mouth, newspaper articles, and television. These methods of contact are reasonably quick, so birth and death rates of analogous epidemic models are negligible and subsequently omitted.

\noindent If $\Delta X(t)=(\Delta I, \Delta S)^T$, its expectation can be expressed as follows:
\begin{equation}\label{eqn:ISR_Exp}
E(\Delta X(t))=
\begin{bmatrix}
   -\frac{\beta}{N}IS \\
   \frac{\beta}{N}IS - \gamma S[S+R]
\end{bmatrix}
\Delta t. \nonumber
\end{equation}
\noindent Next, the covariance matrix can be determined using the expectation:
\begin{equation}
\left.\begin{aligned}
V(\Delta X(t))= E(\Delta X(t)[\Delta X(t)]^T)-E(\Delta X(t))E(\Delta X(t))^T) \\
V(\Delta X(t)) \approx E(\Delta X(t)[\Delta X(t)]^T) \\
V(\Delta X(t))=
\begin{bmatrix}
   \frac{\beta}{N}IS &
   -\frac{\beta}{N}IS \\
   -\frac{\beta}{N}IS &
   \frac{\beta}{N}IS+\gamma S 
\end{bmatrix}
\Delta t. \nonumber
\end{aligned}\right.
\end{equation}
\noindent Note that the covariance matrix is both positive definite and symmetric with square root $B\sqrt{\Delta t}=\sqrt{V}$. Approximating the random vector $X(t+\Delta t)$:
\begin{equation}
X(t+\Delta t)=X(t)+\Delta X(t) \approx X(t)+E(\Delta X(t))+\sqrt{V(\Delta X(t))}.
\end{equation}
\noindent This is an Euler approximation to a system of It\^o standard differential equations and assuming reasonably smooth coefficients, the solution of $X(t)$ converges to
\begin{equation}
\left.\begin{aligned}
\frac{dI}{dt}=-\frac{\beta}{N}IS+B_{11}\frac{dW_1}{dt}+B_{12}\frac{dW_2}{dt} \\
\frac{dS}{dt}=\frac{\beta}{N}IS-\gamma S+B_{21}\frac{dW_1}{dt}+B_{22}\frac{dW_2}{dt},
\end{aligned}\right.
\end{equation}
\noindent where $W_1$ and $W_2$ are independent Wiener processes and $B_{ij}$ is the intensity of fluctuations based on the environment (population and network profile in this case) \cite{allen2008mathematical}. %Stochastic Models
%\chapter{MICRO-MODELS}

\section{Micro vs Macro Models}

\section{Markovian Models}

\section{Hybrid Micro Model} %Micro Models
\chapter{Social Marketing Models} \label{chapter_marketing}
In a typical advertising or marketing model, the marketer sends out constant advertisements to a population to encourage individuals to purchase a product or generally subscribe to the advertisement's suggestion. Over the years, several important models have been presented and utilized toward this end. However, with the advent of social media, a new phenomenon of people converting themselves into marketing agents after viewing one or more initial advertisements has arisen. 

Consider the case of a new gadget, vacation destination, or event (such as sky diving). In many real-life instances, these products and activities take on a life of their own as a ``social craze". If you sees friends and acquaintances pushing the value of these things on social media sites, you would be more inclined to try them out yourself and spread them to other members of your social network. Traditional marketing models do not handle the impact of social media on marketing as effectively as traditional marketing environments and a new approach is desirable. Three popular and fundamental models from marketing are examined in this chapter: the Vidale-Wolfe model, the Bass model, and the Sethi model. Finally, a new social media craze model is proposed.

\section{Vidale-Wolfe Model}
To establish a modern social marketing model, let us first examine the traditional and well established Vidale-Wolfe model. The Vidale-Wolfe model is one of the earliest continuous-time advertising models and is given by
\begin{equation} \label{eqn:VidaleWolfe1}
\left.\begin{aligned}
\frac{dS(t)}{dt}=\beta u(t)[M(t)-S(t)]-\delta S(t),
\end{aligned}\right.
\end{equation}
\noindent where $S$ and $M$ are the brand sales and market size, respectively, and $\delta$ is the rate of brand sale decay when there is no advertising \cite{naik2015marketing}. Simply put, when advertisements target untapped markets, growth occurs and when advertising is absent, growth decreases. 

\section{Bass Model}
As one of the most cited authors in management science, Bass models the interaction between buyers and an untapped market via word of mouth information spread. His model describes the diffusion of new technology, innovations, and products \cite{naik2015marketing}. Bass's differential equation model is specified as:
\begin{equation} \label{eqn:Bass1}
\left.\begin{aligned}
\frac{dN(t))}{dt}=(p+\frac{q}{M}N(t))(M-N(t)),
\end{aligned}\right.
\end{equation}
\noindent where $N$ is the time-dependent cumulative buyers, $p$ is the coefficient of innovation, and $q$ is the coefficient of imitation. Notice that the $(M-N(t))$ term is reminiscent of the buyer-market interaction in the Vidale-Wolfe model.

The Bass model has since been generalized to account for advertising and price inputs with the addition of $F(t)$ on the right-hand side of the differential equation, as follows:
\begin{equation} \label{eqn:Bass2}
\left.\begin{aligned}
\frac{dN(t))}{dt}=(p+\frac{q}{M}N(t))(M-N(t))F(t),
\end{aligned}\right.
\end{equation}\\
\noindent where
\begin{equation}
\left.\begin{aligned}
1-\alpha \{[\frac{\dot{p}(t)}{p(t)}]+\beta[\frac{\dot{a}(t)}{a(t)}]\},
\end{aligned}\right.
\end{equation}
\noindent and $\alpha$ and $\beta$ are the sensitivity parameters for price and advertising price, respectively.

\section{Sethi Model} 
Expanding upon the Vidale-Wolfe model, the Sethi advertising model dynamics take the form of a stochastic differential equation as
\begin{equation} \label{eqn:Sethi1}
\left.\begin{aligned}
dX_t=(rU_t \sqrt{1-X_t}-\delta X_t)dt + \sigma (X_t)dz_t,
\end{aligned}\right.
\end{equation}
\noindent where $X_t$ is the market share at time $t$, $U_t$ is the rate of advertising at time $t$, $r$ is the coefficient of the effectiveness of advertising, $\delta$ is the decay constant, $\sigma (X_t)$ is the diffusion coefficient, and $z_t$ is the Wiener process.

Note, that Equation \ref{eqn:Sethi1} can be expanded via the Sorger \cite{sorger1989competitive} approximation where $\sqrt{1-X_t}\approx (1-X_t)+(1-X_t)X_t$, where the first term is the untapped market effect and the second term is the word-of-mouth effect, leading us to the following expanded form of the Sethi model where the stochastic term is omitted for simplicity:
\begin{equation} \label{eqn:Sethi2}
\left.\begin{aligned}
dX_t=(rU_t (1-X_t)+rU_t(1-X_t)X_t-\delta X_t)dt.
\end{aligned}\right.
\end{equation}

\section{Proposed Social Media Craze Model}
Because none of the current marketing models appear entirely able to adapt to the phenomenon of online social media advertisements that become self-sustaining ``crazes", a new model must be formulated. Naturally, the best place to start in such an endeavor is to examine existing models and see where they succeed and where they fall short of our needs. 

If we apply the Vidale-Wolfe model toward social marketing and set $M=1$ to normalize the ``market" as the total population size of all users within a particular online social media network and set $S$ to be a more general term, $X_t$ for a given social media craze that is to be spread, we can see that
\begin{equation} \label{eqn:CrazeModel1}
\left.\begin{aligned}
dX_t=\beta u(t)[1-X_t]-\delta X_t.
\end{aligned}\right.
\end{equation}

If we now compare the normalized Vidale-Wolf model in Equation \ref{eqn:CrazeModel1} to the Sethi model in Equation \ref{eqn:Sethi2}, we can see that the Sethi model would be inappropriate for an online social media advertising campaign if the goal was to achieve a social craze that was self-sustaining, as term $rU_t(1-X_t)X_t$ will vanish as soon as advertising ends. In an online social media community, individuals continue to discuss and spread a product's information that has been sufficiently advertised, even if it doesn't reach ``social craze" status. 

Returning to the normalized Vidale-Wolfe model, the constant $\beta$ can be subdivided into two components, where $\beta_1$ is the social marketing campaign constant and $\beta_2$ represents what we shall call the ``social craze" constant. The social marketing campaign effectiveness is influenced by active spending and advertising over the network to convince people to join in an activity, buy a product, or support a cause. The social craze constant can be seen as the natural inclination of the social media network to effectively advertise to itself via tweets, posts, and likes once people become exposed to an advertisement and decide to spread and share it with others in the social network. Incorporating these refinements for a social craze scenario, we find that
\begin{equation} \label{eqn:CrazeModel2}
\left.\begin{aligned}
dX_t=\beta_1 u(t)[1-X_t] + \beta_2[1-X_t]-\delta X_t.
\end{aligned}\right.
\end{equation}

It is worth noting that the control $u(t)$ is linked to the advertising campaign effectiveness constant $\beta_1$ because only active advertising can be practically controlled in this model. The whims and virility of social media networks are both difficult to predict for any given group and trying to exercise an effective control over said group must likewise be tailored individually to each network group. This model, however, has a negative-slope decay of the second term and will never reach an online viral state.

Consider a final model in which the second term (the social craze term) increases (in a squared fashion here), as is often observed in a social craze, product, or advertisement that goes viral:
\begin{equation} \label{eqn:CrazeModel3}
\left.\begin{aligned}
dX_t=\beta_1 u(t)[1-X_t] + \beta_2 {X_{t}}^{2}-\delta X_t.
\end{aligned}\right.
\end{equation}

This model allows for advertisement to effect an initial spreading of information via a control, but as the number of spreading agents increases, the spreading becomes self-sustaining provided that the social craze parameter is sufficiently large compared to the decay factor, $\delta$, as people naturally grow bored, as with any information over time.

\subsection{Socio-Equilibrium Threshold}
In order for a social craze to ``take hold" and become self-sustaining, it must be able to maintain itself after exterior and forced advertising ceases. As such, when the control $u(t)$ reaches zero, advertising has ceased, leaving only the second two terms of Equation \ref{eqn:CrazeModel3}, the sum of which must be greater than zero in order to continue spreading the information. It follows that
\begin{equation} \label{eqn:SE_Threshold}
\left.\begin{aligned}
X_t > \frac{\delta}{\beta_2}
\end{aligned}\right.
\end{equation}
\noindent is the threshold for a social-craze type spreading of information to maintain itself after advertising ends. %modified marketing models
\part{CONTROLLING INFORMATION SPREAD}
%\include{Chapter_LiteratureReview_Control} %Literature Review Section (Control)
%\chapter{CONTROL STRATEGIES}

\section{What is Control?}

\section{Information Spread: A ``Disease of the Mind"}

\section{Control and Social Theory} %Control Strategies
%\chapter{TYPES OF CONTROL}

\section{Feedback Control}

\section{Optimal Control}

\section{Other Forms of Control} %Types of Control
\chapter{Control Scenario 1: Advertisement and Social Craze} \label{chapter_control_craze}

\section{Description}
The spread of information via advertisements is critical for most businesses and organizations that rely on popular participation to succeed. If the population is unaware of a product or service, it will not be purchased or otherwise utilized. As such, the goal of advertising information is to spread said information as effectively as possible within the bounds of the spreading organization's time and resources. How does one group advertise to another? By hiring individuals as information spreaders who then spread that information in the form of an ad on television, social media, or billboards. In addition, recent years have seen paid ads in the form of search engine and online shop search priority to ad purchasers. Without said hired advertisements, the single company spreader would simply never gain enough traction in the public eye to spread to any significant amount of customers.

Ideally, an organization could simply pour an infinite amount of control via ad costs and repetition to the public which would absorb the information and never forget it. Eventually, everyone would know about the product or service and it's benefits, giving the best opportunity for the advertisers to gain as many customers as possible. Unfortunately, this ideal situation is rarely if ever seen realistically. Resources are limited by the pocket depth of the advertisers as well as practical constraints. Additionally, customers are likely to forget an advertisement's message over time and several ads must be spread to maintain attention.

But what happens when advertisements cease? The information does not simply disappear from the public consciousness. In the absence of active advertising, especially in a digital social media world, talk of the product or service will endure, perhaps even thrive and grow into what is known commonly as a ``social craze". If a product, service, social movement, or meme can sustain itself without active advertising, despite decay, it would ultimately prove beneficial and cost effective for the original advertiser.

\section{Problem Formulation}
The goal to achieve through control of this scenario is to create a self-sustaining social media craze after sufficient active advertising has commenced. Ideally, the advertisers who wish to start this craze want to spend only as much as is required to make the information spread on its own, without requiring additional advertisement resources. At first glance, the problem appears to closely mirror the Ignorant-Spreader-Recovered (ISR) model presented earlier. Alternatively, a spreading rate model with the addition of a decreasing ``forgetting" factor (as product excitement dwindles for a particular model or iteration) can be developed. Both proposed modelings of the advertisement spread account for the main elements of the problem: spreading the information and a natural tendency of interest in the product to diminish. However, the proposed model for social media crazes fits the scenario nicely. 

Recall that the social craze model from Equation \ref{eqn:CrazeModel3} was presented as
\begin{equation}
\left.\begin{aligned}
dX_t=\beta_1 u(t)[1-X_t] + \beta_2 {X_{t}}^{2}-\delta X_t, \nonumber
\end{aligned}\right.
\end{equation}
\noindent where $\beta_1$ is the social marketing campaign constant, $\beta_2$ represents the social craze constant, $\delta$ is the decay constant, and $u(t)$ is the control, expressed through active advertising initiatives.

Recall also that our goal is to ultimately pass the advertising socio-equilibrium threshold in order to ensure the self-sustainability of the social network craze in the absence of active control. Therefore, active control must be maintained until the spreading of the advertisement has reached enough of the population, which is the threshold point:
\begin{equation}
\left.\begin{aligned}
X_t > \frac{\delta}{\beta_2}. \nonumber
\end{aligned}\right.
\end{equation}
Because we wish to minimize the required expenses of the advertisers and accomplish the goal of achieving social craze status as quickly as possible (in this case, a twenty day advertising campaign), we take the cost function to be:
\begin{equation}
\left.\begin{aligned}
J=\int_{0}^{t_f}(u^2(t)+(x-x_d)^2 + \lambda )dt,
\end{aligned}\right.
\end{equation}
\noindent where $\lambda$ is the time to be optimized and $x_d$ is the desired amount of information spreaders, such that the social craze threshold is crossed and the advertisement spread becomes self-sustaining. At this point, there is no further need for control and the control action $u(t)$ will cease.

This scenario is a prime candidate for an optimal control strategy. There are any number of possible ways to achieve the goal, making the least costly, fastest, or most efficient strategy the preferred one. Other research similarly utilizes optimal control principles to find the best solution for a given need \cite{kachroo2017optimal}. The Hamilton-Jacobi-Bellman (HJB) equation, as the basis of optimal control, will be utilized to achieve an optimal control strategy, $u^*(t)$, representing how much advertising should be put out on social media over time. If no analytic solution is attainable, then the Pontryagin minimization principle will be applied to the problem in order to obtain a numeric solution.

\section{Results and Simulations}
The Hamiltonian was calculated, as follows:
\begin{equation}
\left.\begin{aligned}
H = u^2(t) + (x(t)-x_d)^2 + \lambda +J_x[\beta_1(1-x(t))u(t) + \beta_2 x^2(t) - \delta x(t)].
\end{aligned}\right.
\end{equation}
By differentiating the Hamiltonian, with respect to the control $u(t)$ and setting the result equal to zero, the optimal control action can be found by solving for $u^*(t)$:
\begin{equation}
\left.\begin{aligned}
u^*(t) = \frac{J_x \beta_1 (1-x(t))}{2}.
\end{aligned}\right.
\end{equation}
Finally, the resulting $u^*(t)$ control action was used to determine the Hamilton-Jacobi-Bellman partial differential equation,
\begin{equation}
\left.\begin{aligned}
0 = \frac{J_x^2 \beta_1^2 (1-x(t))^2}{4} + J_x x(t)(\beta_2 x(t)-\delta).
\end{aligned}\right.
\end{equation}

Because there is no clear analytic solution from the resulting HJB partial differential equation, another method is required. The Pontryagin minimization principle was applied to the same dynamics and cost function. The Hamiltonian was calculated using the new method:
\begin{equation}
\left.\begin{aligned}
H = g + p^T[a] = Ru^2(t)+ M(x(t)-x_d)^2 + \lambda + p[\beta_1 u(t)(1-x) + \beta_2 x^2(t) - \delta x(t)],
\end{aligned}\right.
\end{equation}
\noindent where $R$ and $M$ are weight constants for the control and tracking elements of the cost function, respectively. From the Hamiltonian, the state and co-state equations were determined as follows:
\begin{equation}
\left.\begin{aligned}
\dot{x}(t) = \frac{\delta H}{\delta p} = \beta_1 u(t)(1-x(t)) + \beta_2 x^2(t) - \delta x(t)\\
\dot{p}(t) = -\frac{\delta H}{\delta x} = 2M(x_d - x(t)) + p(\beta_1 u(t) - 2 \beta_2 x(t) + \delta).
\end{aligned}\right.
\end{equation}
Again, by differentiating the Hamiltonian with respect to the control $u(t)$ and setting the result equal to zero, the optimal control action was found to be
\begin{equation}
\left.\begin{aligned}
u^*(t) = \frac{-p^* \beta_1(1-x^*(t))}{2R}.
\end{aligned}\right.
\end{equation}
Using the calculated $u^*(t)$, along with the state and co-state equations, the necessary conditions can be expressed as follows:
\begin{equation}
\left.\begin{aligned}
\dot{x}^*(t) = \frac{-p^*(t) \beta_1^2}{2}(x^{*2}(t)-2x^*(t)+1) +\beta_2 x^{*2} - \delta x^*(t)\\
\dot{p}^*(t) = \frac{p^{*2}(t) \beta_1^2}{2}(x^*(t) - 1) - 2p^*(t)\beta_2 x^{*2}(t) - p^{*2}(t) \lambda +2x_d.
\end{aligned}\right.
\end{equation}

Using MATLAB, the boundary value problem was simulated with the \textit{bvp4c} function. The resulting plots of states versus time at various parameters are shown in Figure \ref{fig:Social_Craze_1}, Figure \ref{fig:Social_Craze_2}, Figure \ref{fig:Social_Craze_3}, and Figure \ref{fig:Social_Craze_4}.
\begin{figure}[!htbp] \centering
  \includegraphics[width=0.7\linewidth]{figures/Social_Craze_1_u.eps}
  \caption{Social craze control: $\beta_1=0.1,\beta_2=0.5,d=0.1,x_d=0.2$}
  \label{fig:Social_Craze_1}
\end{figure}

The parameters in Figure \ref{fig:Social_Craze_1} simulate a minor business advertiser whose advertisements only influence a small portion of those who view it. However, the social media group to which the advertisement is targeted is highly interactive and connected. If someone likes the product, they will be very likely to share that information with their friends. It is easy to observe the need for significant control initially, as the social media advertisement system is sustained by the active advertisements. As the system progresses over time, less active advertisement control is needed as more social media ``buzz" is accruing and spreading the advertisement information. Eventually, no active control (and hence, advertisement effort and cost) are required to sustain the social media information. It becomes a ``craze" and is self-sustaining without additional control requirements upon reaching the desired end time. 
\begin{figure}[!htbp] \centering
  \includegraphics[width=0.7\linewidth]{figures/Social_Craze_2_u.eps}
  \caption{Social craze control: $\beta_1=0.6,\beta_2=0.5,d=0.1,x_d=0.2$}
  \label{fig:Social_Craze_2}
\end{figure}

In Figure \ref{fig:Social_Craze_2}, the advertiser has significantly more sway over people through its advertisement actions, as denoted by the $\beta_1$ parameter. All other parameters are identical to the previous case. Here, the advertiser reaches the desired social media craze threshold much faster. In fact, as shown in Figure \ref{fig:Social_Craze_3}, halving the amount of control weight via $R$, gives similar results with less control requirements, which generally translates to less expenditure. 
\begin{figure}[!htbp] \centering
  \includegraphics[width=0.7\linewidth]{figures/Social_Craze_3_u.eps}
  \caption{Social craze control: $R=.5,\beta_1=0.6,\beta_2=0.5,d=0.1,x_d=0.2$}
  \label{fig:Social_Craze_3}
\end{figure}

The impact of the decay factor cannot be understated. Figure \ref{fig:Social_Craze_4} follows the same parameters as the second case, but increases the decay rate, $\delta$. In other words, a popular advertiser is advertising to a receptive group that will readily post and share the product, social cause, etc. However, this group grows disinterested or bored quickly. As a result, significantly more control is required when attempting to form a social media craze out of no initial group information.
\begin{figure}[!htbp] \centering
  \includegraphics[width=0.7\linewidth]{figures/Social_Craze_4_u.eps}
  \caption{Social craze control: $\beta_1=0.6,\beta_2=0.5,d=0.3,x_d=0.6$}
  \label{fig:Social_Craze_4}
\end{figure}

Using the above simulations, it is not difficult to imagine the types of companies that can be similarly modeled and simulated by changing the parameters. Apple, for example, has a strong enough brand following that very little active advertisement is needed to create a group obsession over the latest device. In contrast, governmental health organizations may pour massive amounts of advertising into changing public opinion over bad health habits and achieve very slow progress.
\chapter{Control Scenario 2: Stopping a Fake News Outbreak} \label{chapter_control_fake}

\section{Description}
Consider a scenario similar to China's Salt Panic in which false news is being spread along not only rumor channels, but also trusted internet news sources. The government is aware of the threat that the news may pose to the community and must take steps to prevent it from becoming an information epidemic, as it may cause mass public panic. The fake news of this type is traditionally fast spreading and the damage from it will be done in short order if the information is not quelled. Obviously, this potential information epidemic has already been spreading around the population before the government became aware of its growing popularity.

Luckily, the government has access to modern emergency alert information and official and direct news distribution over the internet. ``Tweets", live YouTube press conferences, and cell phone alerts similar to common ``amber alerts" are all options of quick communication to help stifle the false news. Note that in many ways, it does not matter if the news is real or fake or if the government wishes to spread or diminish the news. The core formulation and strategy of control here are analogous.

\section{Problem Formulation}
The goal of the control action is to ultimately prevent the fake news from taking on a substantial life of its own within the public's social media networks. Recall the concept of herd immunity, where if a sufficient percentage of the population is immunized from the fake news, it will never be able to ``take off" into becoming an information epidemic. The idea of a control condition for regulating a desired property at critical point is a powerful tool in dealing with problems such as this \cite{agarwal2015feedback}. One strategy by which to do this is to educate the population before they receive word of the fake news. This method effectively shrinks the pool from which spreaders of fake news can pull (the ignorant individuals) such that they are unable to create as many future spreaders of the fake news because there are simply not enough members of the ignorant class remaining to convert. With a sufficiently low number of spreaders and, hence, high number of educated ignorants, the fake news information can never reach an epidemic state.

Recall first the basic ISR model, focusing attention on the dynamics of the ignorant and spreader classes, modified here as:
\begin{equation}
\left.\begin{aligned}
\dot{x}_1(t) = -\beta x_1(t)x_2(t) - \beta u(t)x_1(t)\\
\dot{x}_2(t) = \beta x_1(t)x_2(t) - \gamma s(t).
\end{aligned}\right.
\end{equation}
\noindent where $x_1(t)$ and $x_2(t)$ are ignorants and spreaders of the fake news, respectively, and the term $-\beta u(t)x_1(t)$ is the attempt to take away ignorant individuals by educating them on the false nature of the news.
Also recall from herd immunity theory, that the basic reproductive number $R_0$ is essentially the ratio of the spreading rate and stifling rate and that for no information epidemic to occur, we must satisfy
\begin{equation}
\left.\begin{aligned}
R_0 < \frac{1}{1-p},
\end{aligned}\right.
\end{equation}
\noindent where $p$ is the percentage of the population that must be immunized (through education in this case) for the fake news epidemic to not take hold. Since the percent recovered population in the ISR model is equal to $1-I-S$, which is also equal to the desired percentage of group education against the fake news, the equation can be expressed as
\begin{equation}
\left.\begin{aligned}
x_1(t) + x_2(t) < \frac{\gamma}{\beta}.
\end{aligned}\right.
\end{equation}
\noindent which is the control objective.
The Pontryagin minimization method will be used to optimize both the amount of control exerted, the sum of the ignorant and spreading individuals, and time. The cost function is chosen as
\begin{equation}
\left.\begin{aligned}
J=\int_{0}^{t_f}(Ru^2(t)+Q(x_1(t)+x_2(t))^2 + \lambda )dt,
\end{aligned}\right.
\end{equation}
\noindent where $\lambda$ is the time to be optimized and $u(t)$ is the control exercised by public social media posts, announcements, advertisements, text alerts, etc. The constants $R$ and $Q$ are adjustment weight factors to put stronger or weaker emphasis on each element of the cost function.

\section{Results and Simulations}
The Hamiltonian was calculated, as follows:
\begin{equation}
\left.\begin{aligned}
H = (Ru^2(t)+Q(x_1(t)+x_2(t))^2 + \lambda ) +p_1[- \beta x_1(t)x_2(t) - \beta u(t)x_1(t)]\\ + p_2[\beta x_1(t)x_2(t) - \gamma x_2(t)].
\end{aligned}\right.
\end{equation}
The state and co-state equations are calculated from the dynamics, the cost function, and the Hamiltonian:
\begin{equation}
\left.\begin{aligned}
\dot{x}_1(t) = \frac{\delta H}{\delta p_1} = -\beta x_1(t)x_2(t) - \beta u(t)x_1(t)\\
\dot{x}_2(t) = \frac{\delta H}{\delta p_2} = \beta x_1(t)x_2(t) - \gamma s(t)\\
\dot{p}_1(t) = -\frac{\delta H}{\delta x_1(t)} = -2Q(x_1(t)+x_2(t)) + p^*_1 \beta (x_2(t)+u(t)) - p^*_2 \beta x_2(t)\\
\dot{p}_2(t) = -\frac{\delta H}{\delta x_2(t)} = -2Q(x_1(t)+x_2(t)) + p^*_1- p^*_2) \beta x_1(t) + p^*_2 \gamma.
\end{aligned}\right.
\end{equation}
By differentiating the Hamiltonian with respect to the control $u(t)$ and setting the result equal to zero, the optimal control action was found to be:
\begin{equation}
\left.\begin{aligned}
u^*(t) = \frac{-p^*_1 \beta x_1(t))}{2R}.
\end{aligned}\right.
\end{equation}

Using MATLAB, the boundary value problem was simulated with the \textit{bvp4c} function. With initial conditions $x_1(0)$ and $x_2(0)$ as the initial percent of the population ignorant of the fake news and percentage of the population already actively spreading it, respectively. Values for $x_1(t_f)$ and $x_2(t_f)$ will tend towards zero, as the entire population is either educated or stifled from spreading the fake news. Their sum must be below the information epidemic threshold stated in the control objective. The resulting plots of states versus time are shown in pairs for ease of visual understanding, the first displaying the states $x_1(t)$ and $x_2(t)$ and the second displaying $x_1(t)+x_2(t)$, the objective it must achieve, and the control $u(t)$ used to meet the objective.
\begin{figure}[!htbp] \centering
  \includegraphics[width=0.7\linewidth]{figures/Fake_News_1a_u.eps}
  \caption{Fake news control with many spreaders: $x_2(0)=0.30$}
  \label{fig:Fake_News_1a}
\end{figure}
\begin{figure}[!htbp] \centering
  \includegraphics[width=0.7\linewidth]{figures/Fake_News_1b_u.eps}
  \caption{Fake news control with many spreaders: $(x_1(t)+x_2(t))_{desired}<0.25$}
  \label{fig:Fake_News_1b}
\end{figure}

In the first simulated case, shown in Figure \ref{fig:Fake_News_1a} and Figure \ref{fig:Fake_News_1b}, a fake news story with a spreading rate of $\beta=0.004$ and a stifling rate of $\gamma=0.001$ has already been picked up by thirty percent of the population before action is taken to stop it. This is dangerously close to the required twenty-five percent of spreaders needed to achieve an information epidemic state (in which everyone at some point believes in the fake news, however briefly). As such, the most control is required at the earliest time possible. Spending resources on advertisements, paid social media posts, and public service alert texts would be ineffective if the allowable resources were spent at regular intervals instead of being front weighted. Once the total percentage of spreaders begins to decline, then control can be eased considerably. Ultimately, no control is necessary once there are insufficient ignorants left to learn the fake news story and the remaining information spreaders will naturally decay with no receptive audience left.
\begin{figure}[!htbp] \centering
  \includegraphics[width=0.7\linewidth]{figures/Fake_News_2a_u.eps}
  \caption{Fake news control with few spreaders: $x_2(0)=0.10$}
  \label{fig:Fake_News_2a}
\end{figure}
\begin{figure}[!htbp] \centering
  \includegraphics[width=0.7\linewidth]{figures/Fake_News_2b_u.eps}
  \caption{Fake news control with few spreaders: $(x_1(t)+x_2(t))_{desired}<0.25$}
  \label{fig:Fake_News_2b}
\end{figure}

In a second case, shown in Figure \ref{fig:Fake_News_2a} and Figure \ref{fig:Fake_News_2b}, the same story with an identical spreading and stifling rate has only initially spread to ten percent of the population before the government becomes aware of the fake news and decides to take action. Again, the same pool of control resources should be spent as soon as possible, but in this case, there is less urgency since the spreaders begin at a lower level. Resources can be spent less steeply than in the previous case (as shown by $u(t)$). Additionally, the potential fake news epidemic is stopped considerably earlier as one might expect with the spreaders having less initial time to spread the information to ignorants before a control is applied. Depending on the weights placed on the cost function constants, $R$ and $Q$, the problem solution can be modified to optimize total control over quick group immunity to the fake news or strive for a balance between the two. 

If one wished instead to encourage a fake news information epidemic, such that everyone at some point learns it, the problem would be performed similarly, but with a reversed control objective and actively advertising false information to the ignorant class.
%\chapter{CONTROLLING POLITICAL CAMPAIGNS}

\section{Description}

\section{Problem Formulation}

\section{Control Strategy}

\section{Results and Simulations}
\chapter{CONTROL SCENARIO 3: CONTROLLING CONTENTIOUS INFORMATION SPREAD}

\section{Description}

\section{Problem Formulation}

\section{Control Strategy}

\section{Results and Simulations}
%\chapter{OPTIMAL CONTROL STRATEGIES}

\section{What is Optimal Control?}

\section{Formulating the Problem}

\section{Cost Functions}

\section{Our Contribution: Controlling Contentious Information} %Optimal Control Strategies
\chapter{Conclusion}
The overriding purpose of this study was to provide an overview of social media information spread theory, modeling, and analysis, as well as provide some examples techniques on how to control information spread in various scenarios. Multiple mathematical models were proposed to better model social media information spread compared to the spreading of information through traditional channels.

The need and importance of studying information spread over social media was established through the use of modern examples, including political campaigning and fake news. A high level background of information categories was presented, along with the important roles played by ignorants, spreaders, and stiflers within a population as it experiences new information. Popular theories on social networking, particularly with respect to online social media groups, were summarized and some of the widely used modern online social networks were identified.

The groundwork for the technical and objective study of social networks was given, including the classifications of relationship types and the qualities that describe them, such as reciprocity, balance, and the presence or absence of homophily. An overview of the technical aspects of social network structure was exhibited, including mathematical formulas and qualitative descriptions and examples. These technical concepts included density, strength of ties, centrality, distance, cohesion, the adjacency matrix and more. 

Several models were expressed, described, and simulated to demonstrate their usage in a social media information spread context. Deterministic models included various common epidemiology-based information spread models involving ignorant, spreader, and recovered classes, particularly the popular Maki-Thomson model for rumor spread. Due to the growing influence of online social media networks on the spread of information, several new models were proposed to account for differences between traditional information spread and that of modern digital social media. Stochastic models were also presented briefly, explaining the need for stochastic modeling in real-life information spread systems as well as providing some stochastic models based on the previously discussed deterministic models. Aside from traditional mathematical epidemiology based models, three influential social marketing models were summarized and framed in the context of information spread applications. Again, shortcomings with these models when applied to online social media information spread prompted the proposal of a new model in order to describe online social craze phenomenons.

Two scenarios were created in order to explore the potential of creating a social craze using the proposed model and of applying the idea of herd immunity toward a potential fake news outbreak. For both of these scenarios, dynamics and cost functions were established and the Pontryagin minimization principle was applied to achieve optimal control actions. The scenarios were simulated using MATLAB under a variety of different parameters to demonstrate how changes influence the evolution and control of the systems. Potential real-world examples behind the parameter choices were also discussed.

Future work in online social media information spread would include testing of the proposed models using data from Google, Twitter, or Facebook to track the rise and decay of carefully chosen tag words and images. This data can be used to determine general parameters for similar information spreading systems to hopefully create practical and usable online information spread predictions. %Conclusion
\bibliographystyle{IEEEtran}
\bibliography{Michael_References}
%\appendix% Starts chapter numbering with A, B, C, etc.
%\addtocontents{toc}{\noindent Appendices\par}% Puts  "Appendices" in Table of Contents
\chapter*{APPENDIX} %\label{appA}
\noindent \textbf{Ignorant-Spreader Model - 1}
\lstinputlisting[style=Matlab-editor]{code/si.m}
\textbf{Ignorant-Spreader Model - 2}
\noindent \lstinputlisting[style=Matlab-editor]{code/ypsi.m}
\noindent \textbf{Ignorant-Spreader-Ignorant Model - 1}
\lstinputlisting[style=Matlab-editor]{code/ISS.m}
\textbf{Ignorant-Spreader-Ignorant Model - 2}
\noindent \lstinputlisting[style=Matlab-editor]{code/ypISS.m}
\noindent \textbf{Ignorant-Spreader-Recovered Model - 1}
\lstinputlisting[style=Matlab-editor]{code/ISR.m}
\textbf{Ignorant-Spreader-Recovered Model - 2}
\noindent \lstinputlisting[style=Matlab-editor]{code/ypISR.m}
\noindent \textbf{Ignorant-Spreader-Counterspreader-Recovered Model - 1}
\lstinputlisting[style=Matlab-editor]{code/ISCR.m}
\textbf{Ignorant-Spreader-Counterspreader-Recovered Model - 2}
\noindent \lstinputlisting[style=Matlab-editor]{code/ypISCR.m}
\noindent \textbf{Hybrid ISCR Model - 1}
\lstinputlisting[style=Matlab-editor]{code/hybrid_sirc.m}
\textbf{Hybrid ISCR Model - 2}
\noindent \lstinputlisting[style=Matlab-editor]{code/hybrid_ypsirc.m}
\noindent \textbf{ISSRR Model - 1}
\lstinputlisting[style=Matlab-editor]{code/ISSRR.m}
\textbf{ISSRR Model - 2}
\noindent \lstinputlisting[style=Matlab-editor]{code/ypISSRR.m}
\noindent \textbf{Stochastic ISI Model}
\lstinputlisting[style=Matlab-editor]{code/SDE_ISI_Epidemic_Model.m}
\textbf{Social Craze Simulation}
\noindent \lstinputlisting[style=Matlab-editor]{code/Social_Craze_BVP.m}
\textbf{Fake News Simulation}
\noindent \lstinputlisting[style=Matlab-editor]{code/Herd_Immunity_bvp.m}
\addcontentsline{toc}{schapter}{\bibname} %There are some special commands in this file.
\include{Vita}  %The Vita is the last page.

\end{document}