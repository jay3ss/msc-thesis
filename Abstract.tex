\chapter*{ABSTRACT}
\addcontentsline{toc}{schapter}{ABSTRACT}% This command adds the Abstract to the table of contents.

\begin{center}
\textbf{Digital Media, Social Campaigns, and Fake News: Mathematical Modeling and Control Methods}

by

Michael Muhlmeyer

 $\langle$Dr. Shaurya Agarwal$\rangle$, Thesis Advisor\\*[-12pt]%Single spaced.
 Professor of Electrical Engineering \\*[-12pt]
 California State University, Los Angeles  %

 
 \end{center}

The role of information spread and the impact it has on societies in the modern world cannot be understated. In the age of mass communication, digital misinformation, and social media, the importance of understanding and developing control mechanisms for information spread are doubly necessary. While traditional information spread has been examined in detail from a variety of angles over the decades, little attention has been given to the relatively recent phenomena of the super-fast spread of information via social media and the rise and impact of ``fake news" within said information networks. In Part I, a background of information spread theory, terminology, and applications are presented and organized in both a general setting and specifically as information spread applies to social media networks. The importance and influence of a network's structure to the spread of information is also discussed. In Part II, several traditional dynamic models are presented, built upon, and re-framed in the modern context of social media information spread using differential equations. A new model is proposed to address networks and sets of adjacent networks in which information learned and spread is highly polarized, contentious, or unverified. In Part III, control mechanisms and strategies are examined and evaluated along with supporting social theory. These control strategies are developed and applied to sample case studies using the models discussed previously. The results from the proposed control methods of the sample scenarios are calculated, simulated, and discussed.