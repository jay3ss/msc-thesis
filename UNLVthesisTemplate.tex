\documentclass[oneside,12pt]{book}
    % The option oneside is required for UNLV theses.
    % The 12pt is optional -- UNLV will allow 10pt too.
    % 10pt is the default, so to switch to 10pt, just delete 12pt in the above.
\usepackage{amsthm,amssymb,amsmath,graphicx}
\usepackage{etex}
\usepackage{multirow}
\usepackage{graphicx,subfigure}
%\usepackage{graphicx}
\usepackage{pstricks,pst-node,pst-plot,pstricks-add,pst-grad,pst-slpe,overpic}
%\usepackage{showframe} - to show the frame layout
%\usepackage{layout} - to show the page layout
%\usepackage{tikz}
%\usetikzlibrary{arrows,shapes}
\usepackage{amsmath,mathtools,amsthm}
%\usepackage{epstopdf}
\usepackage{caption,verbatim}
%\usepackage[section]{placeins}
%
\newcommand{\sgn}{\mathop{\mathrm{sgn}}}
%\newtheorem{theorem}{Theorem}[section]
%\renewcommand{\thetheorem}{\arabic{section}.\arabic{theorem}}
%\newtheorem{lemma}[theorem]{Lemma}
%\newtheorem{proposition}[theorem]{Proposition}
%\newtheorem{corollary}[theorem]{Corollary}
%\newtheorem{conjecture}[theorem]{Conjecture}
%\newtheorem{definition}[theorem]{Definition}

 % Optional packages.
\usepackage{setspace,myUNLVthesis} % Required packages.  Look at UNLVthesis.sty to see how
    % LaTeX is instructed to set things up.  This file may need some tweaking.
    % setspace.sty is not normally part of MikTeX.  It can be obtained from www.ctan.org.
    % Do a search on setspace.sty.
    % The files setspace.sty and UNLVthesis.sty should be in the same directory
    % as this file (or in a directory of MikTeX where LaTeX will know to find it
    % -- for example, where other style files are).
\pagestyle{unlvplain} %  This is defined in UNLVthesis.sty.  Headings are empty except for page numbers,
    % and the page numbers are the same size as the text.  Most documents use a different size so that
    % it it is difficult to mistake it as part of the text.

%These define the format and numbering of theorem like environments.
\newtheorem{theorem}{Theorem}
\newtheorem{corollary}[theorem]{Corollary}%Corollaries and Lemmas are numbered as theorems.
\newtheorem{lemma}[theorem]{Lemma}

%These define the format and numbering of definition like environments.
\theoremstyle{definition}%This environment is not in italics, like theorems are.
\newtheorem{definition}{Definition}
\newtheorem*{introduction}{Introduction}%The * means it is unnumbered.
\newtheorem*{conclusion}{Conclusion}

% Put definitions here.  For example, suppose you often use the Greek characters
% alpha, beta, etc., which in LaTeX are \alpha, \beta, etc. (in math mode only).
% Then it may be easier to create shortcuts for these commands, such as:
%\def\aa{\alpha}
%\def\bb{\beta}
% Now, instead of typing \alpha, we can type \aa.
% Here's another I often use:
\def\Bbb#1{{\mathbb #1}} % \Bbb is an obsolete command, but I'm old
% and still use it, so I define it to do what it used to do.  The usage
% is like \Bbb R, which will produce a blackboard bold R, and is literally
% translated to {\mathbb R}.  Note that this command includes a single
% argument.

\begin{document}
%\layout{} --- to check the layout margins of the page enable it!!
\pagenumbering{roman}% Items before Chapter one have roman numbers (if any).
\thispagestyle{empty}%This page has no page number.
 \begin{center}
 YOUR TITLE:\\ YOUR SUBTITLE\\*[24pt]% The title can be no more than 80 characters -- UNLV rule.
 %  The command \\*[48pt] means carriage return with a 48pt gap.  These numbers can be adjusted
 %  to improve the look.

\normalsize by\\*[24pt]

Your Name\\*[24pt]

 Bachelor of Science - Your degree\\*[-12pt]%Single space
 Your school, CA, USA\\*[-12pt]
 Your graduation year\\*[12pt]
 
 Master of Science - Electrical Engineering\\*[-12pt]%Single space ...
  California State University Los Angeles, Los Angeles, CA, USA\\*[-12pt]
 2017\\*[36pt]

 A thesis submitted in partial fulfillment\\*[-12pt]
 of the requirements for the\\*[24pt]

 {Master of Science - Electrical Engineering} \\*[24pt]
 
 {Department of Electrical and Computer Engineering}\\*[-12pt]
 {College of Engineering, Computer Science, and Technology}\\*[-12pt]
 {Graduate College}\\*[24pt]
 
 {California State University, Los Angeles}\\*[-12pt]
 {December 2017}
\end{center}
 %Replace this file name with the name of your title page.
    % A copyright statement is optional and would be placed here.
    % The copyright page has no page number -- the title page is always page i and the
    % Thesis Approval page is always page ii.
\newpage \setcounter{page}{3} %The Thesis Approval page is page ii.  It is inserted separately.
\include{SampleAbstract} % This is where the abstract goes.
\include{SampleAcknowledgements}  
\tableofcontents %inserts table of contents
\listoftables \addcontentsline{toc}{schapter}{\listtablename}%
\listoffigures \addcontentsline{toc}{schapter}{\listfigurename}% Comment these out if there are no figures or tables.

%Acknowledgements come after the tables.

\newpage % Do not remove this command.  It's there to make sure the page numbering is correct.
\pagenumbering{arabic} %Chapter 1 begins on page 1.

\include{SampleChapter1_Intro} %Replace these with your chapters.
\include{SampleChapter2_BackGround}
\include{SampleChapter3_LWR}
\include{SampleChapter4_TT}
\include{SampleChapter5_stationary}
\chapter{Conclusion}\label{sec:conclusion}%tles must be capitalized.

%%%%%%%%%%%%%%%%%%%%%%%%%%%%%%%%
This thesis solved two inverse problems using Mean Field Games. In order to derive the classic traffic LWR model based on specific driver behavior cost leading to that model, we identified the costs functions whose solutions through mean field games lead to the derivation of LWR model for traffic. We also derived the travel time spatio-temporal model obtained as a solution to an inverse problem. The paper then discussed the stationary mean field games and solved the two inverse problems numerically for the stationary case.

In our models, we have shown how the microscopic driver behavior leads to the classic Greenshield's fundamental relationship between traffic density and traffic speed, as well as the well known LWR conservation law dynamics for traffic.  We then enhanced the formulation to also show how microscopic driver behavior based on travel time considerations also produce the very significant distributed parameter model for travel time dynamics. The analysis of the stationary versions of the models showed behavior that is consistent with long term expectation of the evolution. 
\bibliographystyle{IEEEtran}
\bibliography{paper}
\addcontentsline{toc}{schapter}{\bibname} %There are some special commands in this file.
%\begin{spacing}{1} %This makes this page single spaced.
\thispagestyle{plain} % This puts the page number at the bottom.
\begin{center}
CURRICULUM VITAE\addcontentsline{toc}{schapter}{CURRICULUM VITAE}\\*[2\baselineskip]% Don't use \chapter* -- the Vita begins
    % at the top of the page.
 Graduate College \\
 California State University, Los Angeles \\*[\baselineskip]
 Michael Muhlmeyer\\*[\baselineskip]
\end{center}

%\noindent Local Address:\\ % Include this if different from your home address.
%***\\*[2\baselineskip]

\noindent Home Address:

 8331 Densmore Ave.

 North Hills, CA 91343\\*[.5\baselineskip]

\noindent Degrees:

 Bachelor of Science, ECE, 2009

 California State University, Northridge, Northridge, CA, USA\\*[.5\baselineskip]

 Master of Science, EE, 2017

 California State University, Los Angeles, Los Angeles, CA, USA\\*[.5\baselineskip]

%\noindent Special Honors and Awards: % Omit, if none.

%\noindent Publications:  % Omit, if none.

%\noindent  Thesis Title: %(Or Thesis title)
%Inverse Problem for Non-viscous Mean Field Control: Example from Traffic\\*[.5\baselineskip]

%\noindent  Thesis Examination Committee: %(Or Thesis ...)

 %Chairperson, Dr.~Monika Neda, Ph.D. %Put ~ after a period if the period should not
    % be interpreted as an end of sentence.  This will restrict how much space can be there,
    % and will prevent a line break there.  This is not necessary if there is no space after
    % the period, for example, in www.unlv.edu

%Co-chair, Dr.~Pushkin Kachroo, Ph.D.
 
 %Committee Member, Dr.~Amei Amei,  Ph.D. % I don't remember anymore ...

 %Committee Member, Dr.~Dieudonne Phanord, Ph.D.

 Graduate Faculty Representative, Dr. Shaurya Agarwal,  Ph.D.

\end{spacing}
  %The Vita is the last page.

\end{document}
