\chapter{Conclusion}
The overriding purpose of this study was to provide an overview of social media information spread theory, modeling, and analysis, as well as provide some examples techniques on how to control information spread in various scenarios. Multiple mathematical models were proposed to better model social media information spread compared to the spreading of information through traditional channels.

The need and importance of studying information spread over social media was established through the use of modern examples, including political campaigning and fake news. A high level background of information categories was presented, along with the important roles played by ignorants, spreaders, and stiflers within a population as it experiences new information. Popular theories on social networking, particularly with respect to online social media groups, were summarized and some of the widely used modern online social networks were identified.

The groundwork for the technical and objective study of social networks was given, including the classifications of relationship types and the qualities that describe them, such as reciprocity, balance, and the presence or absence of homophily. An overview of the technical aspects of social network structure was exhibited, including mathematical formulas and qualitative descriptions and examples. These technical concepts included density, strength of ties, centrality, distance, cohesion, the adjacency matrix and more. 

Several models were expressed, described, and simulated to demonstrate their usage in a social media information spread context. Deterministic models included various common epidemiology-based information spread models involving ignorant, spreader, and recovered classes, particularly the popular Maki-Thomson model for rumor spread. Due to the growing influence of online social media networks on the spread of information, several new models were proposed to account for differences between traditional information spread and that of modern digital social media. Stochastic models were also presented briefly, explaining the need for stochastic modeling in real-life information spread systems as well as providing some stochastic models based on the previously discussed deterministic models. Aside from traditional mathematical epidemiology based models, three influential social marketing models were summarized and framed in the context of information spread applications. Again, shortcomings with these models when applied to online social media information spread prompted the proposal of a new model in order to describe online social craze phenomenons.

Two scenarios were created in order to explore the potential of creating a social craze using the proposed model and of applying the idea of herd immunity toward a potential fake news outbreak. For both of these scenarios, dynamics and cost functions were established and the Pontryagin minimization principle was applied to achieve optimal control actions. The scenarios were simulated using MATLAB under a variety of different parameters to demonstrate how changes influence the evolution and control of the systems. Potential real-world examples behind the parameter choices were also discussed.

Future work in online social media information spread would include testing of the proposed models using data from Google, Twitter, or Facebook to track the rise and decay of carefully chosen tag words and images. This data can be used to determine general parameters for similar information spreading systems to hopefully create practical and usable online information spread predictions.