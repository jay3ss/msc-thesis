\chapter{LITERATURE REVIEW OF INFORMATION SPREAD}% Chapter titles are capitalized

\section{Social Network Information Diffusion}
%Information Diffusion on Online Social Networks
Weng asserts that information spread in online social networks have four main components: the actors that see and spread the message, the content of the message itself, the network topology through which the message much spread, and finally, the processes by which the message diffuses throughout the network topology.
Actors, which are individuals in the social network, are said to have limited attention capacity for information that is cared about. With so much information and messages on the internet, only so much can be deemed important enough to read and spread. Tie strength between the message's origin and the actor, topical interest level, homophily (or ``love of the same") social factors, and social group reinforcement all play roles in whether or not the actor will consume and spread any given set of information. Weng finds that only very strong (close-friend, for example) and very weak actor-actor ties attract the most attention in information diffusion, the strong ones for building and maintaining social relationships and the weak ones for receiving new information.

The role of content includes the topic itself, the sentiment it expresses, the language used in its presentation, and the culture in which the topic is of interest. By mapping memes and other information as nodes or clusters, a ``topic space". New information can be learned, therefore, by finding communities within the information or meme's co-occurrence network. By analyzing these topic spaces, Weng concluded very diverse topics can predict the popularity of a message within a community, while low diversity of messages tend to increase the influence of a single individual spreader. Finally, strongly tied individuals such as close friends are shown to correlate to a diversity of topics and sharing of said topics, while memes and other universally popular posts do not show a spread difference among differently tied groups.

When examining network topology, Weng concluded that the network community's viewpoint influenced information through social reinforcement and homophily to ``trap" information in an out of the community nodes. As a result, early-stage information does not diffuse as an infectious disease as most other models assume, when that diffusion is between node cluster communities. Because social networks are always evolving as information is being spread within the network, diffusion is highly dependent upon the network topology. Maximum-Likelihood Estimation and examination of social-based links from reposted information is used to determine that the concept of triadic closure strongly affects the formation of social links and network evolution along with ``short-cut" links based on the information that is being disbursed through reposts. Additionally, Weng concludes that strategies for following others in information spread varies greatly and linkage behaviors should be categorized and classified based on their network structural characteristics as well as their behavior characteristics in order to perform an analysis. \cite{weng2014information}

