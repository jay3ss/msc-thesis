\chapter{Social Marketing Models} \label{chapter_marketing}
In a typical advertising or marketing model, the marketer sends out constant advertisements to a population to encourage individuals to purchase a product or generally subscribe to the advertisement's suggestion. Over the years, several important models have been presented and utilized toward this end. However, with the advent of social media, a new phenomenon of people converting themselves into marketing agents after viewing one or more initial advertisements has arisen. 

Consider the case of a new gadget, vacation destination, or event (such as sky diving). In many real-life instances, these products and activities take on a life of their own as a ``social craze". If you sees friends and acquaintances pushing the value of these things on social media sites, you would be more inclined to try them out yourself and spread them to other members of your social network. Traditional marketing models do not handle the impact of social media on marketing as effectively as traditional marketing environments and a new approach is desirable. Three popular and fundamental models from marketing are examined in this chapter: the Vidale-Wolfe model, the Bass model, and the Sethi model. Finally, a new social media craze model is proposed.

\section{Vidale-Wolfe Model}
To establish a modern social marketing model, let us first examine the traditional and well established Vidale-Wolfe model. The Vidale-Wolfe model is one of the earliest continuous-time advertising models and is given by
\begin{equation} \label{eqn:VidaleWolfe1}
\left.\begin{aligned}
\frac{dS(t)}{dt}=\beta u(t)[M(t)-S(t)]-\delta S(t),
\end{aligned}\right.
\end{equation}
\noindent where $S$ and $M$ are the brand sales and market size, respectively, and $\delta$ is the rate of brand sale decay when there is no advertising \cite{naik2015marketing}. Simply put, when advertisements target untapped markets, growth occurs and when advertising is absent, growth decreases. 

\section{Bass Model}
As one of the most cited authors in management science, Bass models the interaction between buyers and an untapped market via word of mouth information spread. His model describes the diffusion of new technology, innovations, and products \cite{naik2015marketing}. Bass's differential equation model is specified as:
\begin{equation} \label{eqn:Bass1}
\left.\begin{aligned}
\frac{dN(t))}{dt}=(p+\frac{q}{M}N(t))(M-N(t)),
\end{aligned}\right.
\end{equation}
\noindent where $N$ is the time-dependent cumulative buyers, $p$ is the coefficient of innovation, and $q$ is the coefficient of imitation. Notice that the $(M-N(t))$ term is reminiscent of the buyer-market interaction in the Vidale-Wolfe model.

The Bass model has since been generalized to account for advertising and price inputs with the addition of $F(t)$ on the right-hand side of the differential equation, as follows:
\begin{equation} \label{eqn:Bass2}
\left.\begin{aligned}
\frac{dN(t))}{dt}=(p+\frac{q}{M}N(t))(M-N(t))F(t),
\end{aligned}\right.
\end{equation}\\
\noindent where
\begin{equation}
\left.\begin{aligned}
1-\alpha \{[\frac{\dot{p}(t)}{p(t)}]+\beta[\frac{\dot{a}(t)}{a(t)}]\},
\end{aligned}\right.
\end{equation}
\noindent and $\alpha$ and $\beta$ are the sensitivity parameters for price and advertising price, respectively.

\section{Sethi Model} 
Expanding upon the Vidale-Wolfe model, the Sethi advertising model dynamics take the form of a stochastic differential equation as
\begin{equation} \label{eqn:Sethi1}
\left.\begin{aligned}
dX_t=(rU_t \sqrt{1-X_t}-\delta X_t)dt + \sigma (X_t)dz_t,
\end{aligned}\right.
\end{equation}
\noindent where $X_t$ is the market share at time $t$, $U_t$ is the rate of advertising at time $t$, $r$ is the coefficient of the effectiveness of advertising, $\delta$ is the decay constant, $\sigma (X_t)$ is the diffusion coefficient, and $z_t$ is the Wiener process.

Note, that Equation \ref{eqn:Sethi1} can be expanded via the Sorger \cite{sorger1989competitive} approximation where $\sqrt{1-X_t}\approx (1-X_t)+(1-X_t)X_t$, where the first term is the untapped market effect and the second term is the word-of-mouth effect, leading us to the following expanded form of the Sethi model where the stochastic term is omitted for simplicity:
\begin{equation} \label{eqn:Sethi2}
\left.\begin{aligned}
dX_t=(rU_t (1-X_t)+rU_t(1-X_t)X_t-\delta X_t)dt.
\end{aligned}\right.
\end{equation}

\section{Proposed Social Media Craze Model}
Because none of the current marketing models appear entirely able to adapt to the phenomenon of online social media advertisements that become self-sustaining ``crazes", a new model must be formulated. Naturally, the best place to start in such an endeavor is to examine existing models and see where they succeed and where they fall short of our needs. 

If we apply the Vidale-Wolfe model toward social marketing and set $M=1$ to normalize the ``market" as the total population size of all users within a particular online social media network and set $S$ to be a more general term, $X_t$ for a given social media craze that is to be spread, we can see that
\begin{equation} \label{eqn:CrazeModel1}
\left.\begin{aligned}
dX_t=\beta u(t)[1-X_t]-\delta X_t.
\end{aligned}\right.
\end{equation}

If we now compare the normalized Vidale-Wolf model in Equation \ref{eqn:CrazeModel1} to the Sethi model in Equation \ref{eqn:Sethi2}, we can see that the Sethi model would be inappropriate for an online social media advertising campaign if the goal was to achieve a social craze that was self-sustaining, as term $rU_t(1-X_t)X_t$ will vanish as soon as advertising ends. In an online social media community, individuals continue to discuss and spread a product's information that has been sufficiently advertised, even if it doesn't reach ``social craze" status. 

Returning to the normalized Vidale-Wolfe model, the constant $\beta$ can be subdivided into two components, where $\beta_1$ is the social marketing campaign constant and $\beta_2$ represents what we shall call the ``social craze" constant. The social marketing campaign effectiveness is influenced by active spending and advertising over the network to convince people to join in an activity, buy a product, or support a cause. The social craze constant can be seen as the natural inclination of the social media network to effectively advertise to itself via tweets, posts, and likes once people become exposed to an advertisement and decide to spread and share it with others in the social network. Incorporating these refinements for a social craze scenario, we find that
\begin{equation} \label{eqn:CrazeModel2}
\left.\begin{aligned}
dX_t=\beta_1 u(t)[1-X_t] + \beta_2[1-X_t]-\delta X_t.
\end{aligned}\right.
\end{equation}

It is worth noting that the control $u(t)$ is linked to the advertising campaign effectiveness constant $\beta_1$ because only active advertising can be practically controlled in this model. The whims and virility of social media networks are both difficult to predict for any given group and trying to exercise an effective control over said group must likewise be tailored individually to each network group. This model, however, has a negative-slope decay of the second term and will never reach an online viral state.

Consider a final model in which the second term (the social craze term) increases (in a squared fashion here), as is often observed in a social craze, product, or advertisement that goes viral:
\begin{equation} \label{eqn:CrazeModel3}
\left.\begin{aligned}
dX_t=\beta_1 u(t)[1-X_t] + \beta_2 {X_{t}}^{2}-\delta X_t.
\end{aligned}\right.
\end{equation}

This model allows for advertisement to effect an initial spreading of information via a control, but as the number of spreading agents increases, the spreading becomes self-sustaining provided that the social craze parameter is sufficiently large compared to the decay factor, $\delta$, as people naturally grow bored, as with any information over time.

\subsection{Socio-Equilibrium Threshold}
In order for a social craze to ``take hold" and become self-sustaining, it must be able to maintain itself after exterior and forced advertising ceases. As such, when the control $u(t)$ reaches zero, advertising has ceased, leaving only the second two terms of Equation \ref{eqn:CrazeModel3}, the sum of which must be greater than zero in order to continue spreading the information. It follows that
\begin{equation} \label{eqn:SE_Threshold}
\left.\begin{aligned}
X_t > \frac{\delta}{\beta_2}
\end{aligned}\right.
\end{equation}
\noindent is the threshold for a social-craze type spreading of information to maintain itself after advertising ends.